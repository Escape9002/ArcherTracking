Mit Bluetooth 5.0 wurde eine neue übertragunsweise zu 
Bluetooth hinzugefügt. Diese nennt sich Bluetooth-Low-Energy und zeichnet
sich durch einen geringen Stromverbrauch und damit einem höherem 
Datendurchsatz aus. \\
\\
Bluetooth sendet Daten in Paketen. Hierbei ist bei Bluetooth-Low-Energy
(zukünftig BLE) der Sender als Server ausgewiesen und der Empfänger als Client.\\
\\
Hierdurch kann der Client bis zu??? Server abfragen ohne sich mit diesen 
verbinden zu müssen. Jeder Server kann unbegrenzt Characteristiken anbieten,
diese Stellen verschieden Datensätze dar, die vom Client abgefragt werden können.\\
\\
Laut Dokumentation beträgt der Maximale Datensatz 244 Bytes pro Paket bei 
aktiviertem DLE. Diese Funktion ließ ich ausgeschaltet, wodurch ich Maximal
27 Bytes pro Paket versenden kann. \\

\section{Datendurchsatz per BLE}
%Etwas stimmt hier nicht an der Rechnung, Fehlersuche!
%Quelle:
%https://www.novelbits.io/bluetooth-5-speed-maximum-throughput/
%--------------------------------------------------------------
2Mbps, steht in Prozessor Doku, lieber nochmal in BLELib nachlesen!
\\
Das Sendeprotokoll von Bluetooth schreibt vor, das ein Datenpaket von
leeren Datenpakten eingepackt wird, somit beträgt die Sendezeit pro 
Datenpaket:

\begin{equation} 
$$
Zeit = Leer + IFS + Data + IFS
Leer = leerespacket(größe) / datarate
$$
\end{equation}

Für mich heißt das:
\begin{equation} 
    $$
leerGröße = 2 + 4 + 2 + 3 = 11 Bytes == 88 bits
$$
\end{equation}

und die Sendezeit für das leere Paket beträgt damit:

\begin{equation}
    $$
leerZeit = 88/2Mpbs = 44 Mikro Sekunden.
$$
\end{equation}


Für ein volles Datenpaket brauche ich:
\begin{equation}
    $$
Voll = 44 + 2*150+ 2+4+2+4+20+3 = 44+300+35 = 379Bytes *8 ==3032 Bits
Vollzeit = 3032 / 2 = 1,516 Mikrosekunden
$$
\end{equation}

Für ein gesaamtes Datenpaket brauche ich damit mindestens:
\begin{equation}
    $$
Zeit = 44 + 2*150 + 1,516 = 345,516 Mikrosekunden
88/2 + 2*150 + (2 + 4 + 2 + 4 + 20 + 3)*8/2
$$
\end{equation}

%-----------------------------------------------------------------
\section{Datengröse}
Die Daten werden als String versendet, diese werden von Arduino mit einer
Null Terminiert. Die Größer der Sensordaten beträgt:
\begin{equation}
    $$
Vorkommastellen (3) + Komma (1) + Dezimalstellen (2) + Terminierung (1) = 7 Char
$$
\end{equation}
1 Char entspricht 1 Byte, somit gilt:
\begin{equation}$$
9 Sensoren * 7 Byte = 63 Byte
63 Byte / 27 Byte = 2,3 Datenpakete pro alle Sensoren
$$
\end{equation}
Somit brauche ich für das Senden aller Sensoren mindestens 3 Characteristics.

\section{Tatsächliche Übetragunsgeschwindigkeit}
%Führe Test durch!
