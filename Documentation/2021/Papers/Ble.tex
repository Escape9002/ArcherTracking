Mit Bluetooth 5.0 wurde eine neue übertragunsweise zu 
Bluetooth hinzugefügt. Diese nennt sich Bluetooth-Low-Energy und zeichnet
sich durch einen geringen Stromverbrauch und dennoch einem höherem 
Datendurchsatz aus. \\
\\
Bluetooth sendet Daten in Paketen. Hierbei ist bei Bluetooth-Low-Energy
(zukünftig BLE) der Sender als Server ausgewiesen und der Empfänger als Client.\\
\\
Die Server bieten ``Services'' an die mit ``Characteristics'' befüllt sind.
So bietet mein Arduino den Service ``MPU9250'' an mit dem Characteristis ``Accl'', ``Gyro''
und ``Mag''.\\
Der Nachteil dieser Verteilung der einzelnen Daten besteht hierbei in der Zeit die für die
Abfrage gebraucht wird. Jede Characteristic muss einzeln abgefragt werden, hierbei kann ein
Großteil des Datensatzes des MPU9250 verloren gehen.\\
\\
Laut Dokumentation beträgt der Maximale Datensatz von BLE 244 Bytes pro Paket bei 
aktiviertem DLE. Diese Funktion ließ ich ausgeschaltet, wodurch ich Maximal
27 Bytes pro Paket versenden kann. Dieses Problem erklärt ebenfalls weshalb die Sensor-Daten
auf verschiedene Characteristics aufgeteilt werden. Alle Daten passen nicht in ein einzelnes
zu versendendes Paket. \\
\\

\section{Datengröse}
Die Daten werden als String versendet, diese werden von Arduino mit einer
Null Terminiert. \\
\\
Die Größer der Sensordaten beträgt:\\
\textit{Vorkommastellen (3) + Komma (1) + Dezimalstellen (2) + Terminierung (1) = 7 Char}\\
\\
1 Char entspricht 1 Byte, somit gilt:\\
\textit{
9 Sensoren * 7 Byte = 63 Byte \\
63 Byte / 27 Byte = 2,3 Datenpakete pro alle Sensoren
}\\
\\
Somit brauche ich für das Senden aller Sensoren mindestens 3 Characteristics.


\section{Datendurchsatz per BLE}
%Quelle:
%https://www.novelbits.io/bluetooth-5-speed-maximum-throughput/
%--------------------------------------------------------------

Das Sendeprotokoll von Bluetooth schreibt vor, das ein Datenpaket von
leeren Datenpaketen eingepackt wird, ebenso ist eine kurze Wartezeit vorgeschrieben. 
Diese beträgt 150 Mikrosekunden und wird abgekürt mit IFS.
Der Arduino Nano unterstützt 2Mbps bei der BLE-Übertragung, dies ist also die Datenrate.
Des weiteren wird nicht auf eine Antwort des \textit{Client´s} gewartet, was die 
Übertragungsgeschwindigkeit weiter erh"oht.\\
\\
Somit beträgt die optimale Sendezeit pro Datenpaket:\\
\textit{Zeit = Sendedauer[Leer] + IFS + Sendedauer[Voll] + IFS\\
Sendedauer[Leer] = Leerespacket / Datarate}\\
\\
Für mich heißt das:\\
\textit{Leerespacket = 2 + 4 + 2 + 3 = 11 Bytes * 8 = 88 bit}\\
\\
und die Sendezeit für das leere Paket beträgt damit:\\
\textit{Sendedauer[Leer] = 88 bit / 2Mpbs = 44 Mikro-Sekunden}\\
\\
Für ein volles Datenpaket brauche ich:\\
\textit{2+4+2+4+27+3 = 42 Byte * 8 = 336 bit\\
Sendedauer[Voll] = 336 bit / 2Mbps = 168 Mikro-Sekunden}\\
\\
Für ein gesamtes Datenpaket brauche ich somit mindestens:\\
\textit{Zeit = 44 + 150 + 168 + 150 = 512 Mikro-Sekunden}\\
\\
beziehungsweise 0,512 Millisekunden. Die maximal erreichbare Datenübetragungs-Frequenz
liegt bei 512Hz.

%-----------------------------------------------------------------

\section{Tatsächliche Übetragunsgeschwindigkeit}
Die ausgerechnete Datenrate kann in der Praxis kaum erreicht werden, weshalb ein Test zur 
tats"achlichen Datenrate Pflicht ist. Gemessen wurde die Wartezeit am Handy zwischen zwei
Datenpaketen. Da momentan zwei \textit{Characteristic´s} abgefragt werden multipliziere ich die 
Zeit mit zwei. So erhalte ich die Frequenz für ein einziges Datenpaket.
