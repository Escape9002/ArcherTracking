Mit Bluetooth 5.0 wurde eine neue übertragunsweise zu 
Bluetooth hinzugefügt. Diese nennt sich Bluetooth-Low-Energy und zeichnet
sich durch einen geringen Stromverbrauch und dennoch einem höherem 
Datendurchsatz aus. \\
\\
Bluetooth sendet Daten in Paketen. Hierbei ist bei Bluetooth-Low-Energy
(zukünftig BLE) der Sender als Server ausgewiesen und der Empfänger als Client.\\
\\
Die Server bieten ``Services'' an die mit ``Characteristics'' befüllt sind.
So bietet mein Arduino den Service ``MPU9250'' an mit dem Characteristis ``Accl'', ``Gyro''
und ``Mag''.\\
Der Nachteil dieser Verteilung der einzelnen Daten besteht hierbei in der Zeit die für die
Abfrage gebraucht wird. Jede Characteristic muss einzeln abgefragt werden, hierbei kann ein
Großteil des Datensatzes des MPU9250 verloren gehen.\\
\\
Laut Dokumentation beträgt der Maximale Datensatz von BLE 244 Bytes pro Paket bei 
aktiviertem DLE. Diese Funktion ließ ich ausgeschaltet, wodurch ich Maximal
27 Bytes pro Paket versenden kann. Dieses Problem erklärt ebenfalls weshalb die Sensor-Daten
auf verschiedene Characteristics aufgeteilt werden. Alle Daten passen nicht in ein einzelnes
zu versendendes Paket. \\
\\

\section{Datengröse}
Die Daten werden als String versendet, diese werden von Arduino mit einer
Null Terminiert. Die Größer der Sensordaten beträgt:
\begin{equation}
    $$
Vorkommastellen (3) + Komma (1) + Dezimalstellen (2) + Terminierung (1) = 7 Char
$$
\end{equation}
1 Char entspricht 1 Byte, somit gilt:
\begin{equation}$$
9 Sensoren * 7 Byte = 63 Byte \\
63 Byte / 27 Byte = 2,3 Datenpakete pro alle Sensoren
$$
\end{equation}
Somit brauche ich für das Senden aller Sensoren mindestens 3 Characteristics.


\section{Datendurchsatz per BLE}
%Etwas stimmt hier nicht an der Rechnung, Fehlersuche!
%Quelle:
%https://www.novelbits.io/bluetooth-5-speed-maximum-throughput/
%--------------------------------------------------------------
2Mbps, steht in Prozessor Doku, lieber nochmal in BLELib nachlesen!
\\
Das Sendeprotokoll von Bluetooth schreibt vor, das ein Datenpaket von
leeren Datenpakten eingepackt wird, somit beträgt die Sendezeit pro 
Datenpaket:

\begin{equation} 
$$
Zeit = Leer + IFS + Data + IFS
Leer = leerespacket(größe) / datarate
$$
\end{equation}

Für mich heißt das:
\begin{equation} 
    $$
leerGröße = 2 + 4 + 2 + 3 = 11 Bytes == 88 bits
$$
\end{equation}

und die Sendezeit für das leere Paket beträgt damit:

\begin{equation}
    $$
leerZeit = 88/2Mpbs = 44 Mikro Sekunden.
$$
\end{equation}


Für ein volles Datenpaket brauche ich:
\begin{equation}
    $$
Voll = 44 + 2*150+ 2+4+2+4+20+3 = 44+300+35 = 379Bytes *8 ==3032 Bits
Vollzeit = 3032 / 2 = 1,516 Mikrosekunden
$$
\end{equation}

Für ein gesamtes Datenpaket brauche ich damit mindestens:
\begin{equation}
$$
Zeit = 44 + 2*150 + 1,516 = 345,516 Mikrosekunden
88/2 + 2*150 + (2 + 4 + 2 + 4 + 20 + 3)*8/2
$$
\end{equation}

%-----------------------------------------------------------------

\section{Tatsächliche Übetragunsgeschwindigkeit}
%Führe Test durch!
