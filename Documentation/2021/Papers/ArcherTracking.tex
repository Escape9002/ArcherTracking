\documentclass[12pt,titlepage]{scrreprt}
\usepackage[ngerman]{babel}
\usepackage[utf8]{inputenc}
\usepackage{color}
\usepackage{float}
\usepackage[a4paper,lmargin={2.5cm},rmargin={2.5cm},tmargin={2.5cm},bmargin = {2cm},footskip={1cm}]{geometry}
\usepackage{amssymb}
\usepackage{amsthm}
\usepackage{graphicx}
\usepackage{subfig}
\usepackage{wrapfig}
\usepackage{calc}
\usepackage{overcite} 
\usepackage{amsmath}
\usepackage{amssymb}
\usepackage{tabularx}
\usepackage{siunitx}
\usepackage[official]{eurosym}
\usepackage{url}
\renewcommand\citeform[1]{[#1]}
\usepackage{hyperref}
\hypersetup{
    colorlinks=true,
    linkcolor=blue,
    filecolor=magenta,      
    urlcolor=cyan,
    pdftitle={Overleaf Example},
    pdfpagemode=FullScreen,
    }

% !TEX program = xelatex
\begin{document}
% \include{Title}
\begin{titlepage}

	

\title{Archer-Tracking}
\subtitle{Bewegungstracking für Sportler}
\titlehead{\centering\includegraphics[width=15cm]{Bilder/IMG_9973 (4)}}


\author{Antonio Rehwinkel}

\publishers{Unterstützende Lehrer: Herr Czernohous \\ Herr Dierle \\  \texttt{Emails?} \\
\vspace*{2ex} und Professoren: Hochschule1 \\ Hochschule2\\ \texttt{Emails?}}
%- \\ Schiller-Gymnasium Offenburg}

\maketitle

\end{titlepage}
\tableofcontents
\chapter[Vorwort]{Vorwort}

Mithilfe eines Beschleunigungssensoren kann man viele 
Bewegungen erforschen und vermessen.
Mit meinem System sollen mehrere Beschleunigungssensoren 
dazu eingesetzt werden können, Bewegungsabläufe aufzunehmen, 
miteinander zu vergleichen und zu erkennen.\\
\\
Die Daten werden über Bluetooth-Low-Energy an ein Handy 
geschickt, wo Sie sowohl gespeichert als auch verwertet 
werden können. Als Beispiel gilt hier für mich das Bogenschießen, 
bei dem selbst kleine Bewegungen immer wieder auf gleiche Weise 
ausgeführt werden müssen. Mit meinen Sensor sollen hier teure 
Kamerasysteme abgeschafft werden und es so jeden ermöglichen, 
selbst ohne Bogen oder Trainer bei sich Zuhause zu den 
Bewegungsablauf zu trainieren.\\

\section{IMU vs Kameratracking}
Mein Projekt überschneidet sich in seinen Zielen häufig 
mit Tracking das bei VR-Brillen eingesetzt wird. Hier wird
zur Feststellung der Position des Spielers häufig eine 
Kombination aus Kameratracking und Infrarot-LED. \\
Hierbei muss der Spieler die Fernbedienungen festhalten die
die Infrarot-LEDs beinhalten, während die Kameras im Raum 
so verteilt werden müssen das der Spieler immer erkannt wird.\\
Die Neigung des Kopfes und der Hände werden auch hier häufig 
Mithilfe eines IMU bestimmt.\\
\\
Da diese Systeme viel Platz benötigen, viel Geld kosten und 
für die Bildverarbeitung häufig eine große Rechenkraft benötigen
ist dieses System nicht für viele Privatnutzer sinnvoll oder 
bieten einen Bewegungsfreiraum der Sport zu lässt.\\


%\\
%Die IMU-Sensoren bestehen mindestens aus einem Gyroskop und einem 
%Beschleunigungssensor, manche bieten sogar ein Magnetometer an.
%Somit sollte es möglich sein, über die
%Beschleunigung die Distantz die ein Körper mit diesem Sensor zurück
%legt zu messen. Die Neigung und Orientation sind über das Gyroskop 
%und Magnetometer sehr genau messbar.\\
%\\
%Die Vorteile der IMU liegen auf der Hand, sie sind günstig, klein
%und leicht. Aus diesen Gründen trägt fast jeder heutzutage 
%so einen Sensor bei sich, die meisten Handys haben ihn schon eingebaut.\\
%\\
%Für mein Projekt benutze ich dennoch einen eigenen MPU, um die
%Qualität der Messdaten sicher zu stellen.


\section{Bewegungen und Analyse}
Eine Interessante Bewegung stellt vor allem der Zugarm des 
Schüzten dar. Der Auszug verläuft nahezu linear, der häufigere Fehler 
an dieser Stelle versteckt sich allerdings in der Höhe des Zugarms. 
Um diese zu messen muss man die Erdanziehungskraft der Z-Achse herausrechnen.\\
\\
Da der Schussablauf eines Bogenschützen viele Stationen mit verschiedenen Bewegungen 
beinhaltet fällt es häufig sogar den Trainern schwer zwischen einem technisch guten
oder schlechtem Schuss zu unterscheiden.\\
Die Datenlage aus dem 9DOF-System des verwendeten MPU erzeugt eine Datenwolke, die 
diesen Vorgang fürs erste verkompliziert.\\
Die Daten sind allerdings sehr gut zu vergleichen und zu mitteln. Mit diesen 
Eigenschaften kann man über künstliche Intelligenz, genauer, Machine Learning 
die Schüsse klassifiezieren. So kann jeder Schütze seine eigene Datenbasis erstellen,
nach der sein Schuss klassifiziert und so eingeordnet werden können.\\
Ein Vergleich mit einer deutlich größeren Datenbasis als einem einzelnen Schützen ist denkbar.\\
\\
Für dieses System ist es nahezu unwichtig wie viele Körperteile überwacht werden.
Mit mehr Sensoren (oder Vergleichspunkten) wird es lediglich schwieriger für den
Schützen einen guten Wert bei der klassifiezierung zu erreichen.\\
Die Verwendung setzt natürlich vor allem korrekte, genau und viele Daten vorraus,
dies gilt für das Training des Modells so wie für die Live-Daten des Schützen.\\
\\


%\chapter{IMU vs Kameratracking}
%Mein Projekt überschneidet sich in seinen Zielen häufig 
mit Tracking das bei VR-Brillen eingesetzt wird. Hier wird
zur Feststellung der Position des Spielers häufig eine 
Kombination aus Kameratracking und Infrarot-LED. \\
Hierbei muss der Spieler die Fernbedienungen festhalten die
die Infrarot-LEDs beinhalten, während die Kameras im Raum 
so verteilt werden müssen das der Spieler immer erkannt wird.\\
Die Neigung des Kopfes und der Hände werden auch hier häufig 
Mithilfe eines IMU bestimmt.\\
\\
Da diese Systeme viel Platz benötigen, viel Geld kosten und 
für die Bildverarbeitung häufig eine große Rechenkraft benötigen
ist dieses System nicht für viele Privatnutzer sinnvoll oder 
bieten einen Bewegungsfreiraum der Sport zu lässt.\\


%\\
%Die IMU-Sensoren bestehen mindestens aus einem Gyroskop und einem 
%Beschleunigungssensor, manche bieten sogar ein Magnetometer an.
%Somit sollte es möglich sein, über die
%Beschleunigung die Distantz die ein Körper mit diesem Sensor zurück
%legt zu messen. Die Neigung und Orientation sind über das Gyroskop 
%und Magnetometer sehr genau messbar.\\
%\\
%Die Vorteile der IMU liegen auf der Hand, sie sind günstig, klein
%und leicht. Aus diesen Gründen trägt fast jeder heutzutage 
%so einen Sensor bei sich, die meisten Handys haben ihn schon eingebaut.\\
%\\
%Für mein Projekt benutze ich dennoch einen eigenen MPU, um die
%Qualität der Messdaten sicher zu stellen.

\chapter{Hardware}
\section{Arduino Nano 33 BLE}
Der Arduino Nano 33 BLE wurde mit einem M4-ARM-Processor und einem 5.0 
Bluetooth Modul ausgestattet. Damit ist er BLE-Fähig und liefert einen starken
Prozessor für Kommarechnungen. \\
%Tabelle der Daten einfügen
\\
Der Arduino eignet sich durch seine BLE-Fähigkeit und I2C beziehungsweise
SPI-Anschlüsse zur Verwendung mit dem Verwendetem MPU9250. Die geringe Größe
und Stromverbrauch sind weitere Pluspunkte.

\section{MPU9250}
Der benutze IMU in diesem Projekt ist ein Multi-Chip mit einem
3-Achsen Gyroskop, 3-Achsen Beschleunigungssensor und einem 3-Achsen Magnetometer.
Alle Sensoren wurden noch in der Firma kalibriert. Der Chip bietet einen
eingebauten Low-Pass-Filter der eine erste Bearbeitung der Daten vornimmt 
und so den Hauptprozessor entlastet.\\
In der folgenden Tablle ist die Empfindlichkeit und Genauigkeit der 
einzelnen Sensoren notiert, sowie die Übetragunsart und Geschwindigkeit.\\
%Tabelle
\\
Für eine einfache Handhabung benutze ich die I2C Verbindung zwischen 
Arduino und MPU9250.

\section{Stromverbrauch}
Da dieses Jahr fest steht, welche Daten gesendet werden können, lohnt es 
sich nun auch, die benötigte Leistung zu ermitteln.
Diese wurde mit einem Multimeter am Batterieanschluss in
verschiedenen Modi gemessen. Die Ergebnisse stehen in der Tabelle:\\
%Tabelle: Werte: 
%200mA Auslösung:
%BLE aus: 5
%BLE an: 10.2
%BLE Verbunden: 13
%MPU9250 an: 5
%Bereich: 200 | Auflösung : 0,1 uA | +-1% des Messwerts +- 2 Ziffern
%Quelle:https://github.com/Escape9002/ArcherTracking/issues/5
\\
\begin{tabularx}{0.8\textwidth}{l|X|XX}
    Modul & An/Aus & Verbrauch(in mA) \\
    \hline
    BLE & aus & 5 \\
    \hline
    BLE & an & 10.2 \\
    \hline
    BLE & an und verbunden & 13 \\
    \hline
    MPU9250 & an & 5 \\
\end{tabularx}\\
\\
Der Stromverbrauch lässt sich so berechnen und erleichtert die korrekte Batterie-Wahl.
Die verwendeten Formeln: 
\begin{equation}
    $$
    P = U * I \\
    Für die Einheiten gilt nach Multiplikation mit der Zeit:\\
    Ah * V = Wh \\
    Wh / W = h $\rightarrow$ (U*I*t) / U*I = t \\
    $$
\end{equation}
So hat der Arduino eine Leistungsaufnahme von:
\begin{equation}
    $$
    0,005 A * 7,4 V = 0,037 W
    $$
\end{equation}
\\
Die verschiedenen Batterien-Typen stehen in der folgenden Tabelle:\\
\\
\begin{tabularx}{0.8\textwidth}{l|X|X|X|XX}
    Typ & Laufzeit(in Stunden) & Gewicht & Cut-Off-Spannung & Differenz\footnote{zwischen Cut-Off-Spannung und angebotener Spannung}\\
    \hline
    9V & 40 & 50g &7.2V & $9-7.2 = 2,8$\\
    \hline
    CR2025 & 12 & 2,5g & 2V & $3-2 = 1$\\
    \hline
    2 * CR2025 & 48,6 & 5g & 2V & $6-2 = 4$\\
\end{tabularx}\\
%Tabelle
%9V : 40h, 0,05 KG, CutoffVol: 7.2V (9-7.2 = 2,8)
% CR2025 : 12h , 0,0025 Kg, CutoffVolt: 2V (3-2 = 1)
% CR2025 * 2 = 48,6 h, 0,005 Kg, CutoffVolt: 2V (6-2 = 4)
\\
Um eine Stromversorgung des Arduinos sicher zu stellen, benötigt man 2 Knopfzellen
des Typs CR2025. Dennoch hat die Knopfzelle CR2025 nicht nur eine bessere 
Laufzeit sondern ebenfalls weniger Gewicht, braucht weniger Platz und liefert eine bessere 
Differenz zwischen Cut-Off-Spannung zu angebotener Spannung.
\\
Gegen das Umrüsten auf die Knopfbatterie spricht einzig der Umweltschutz.
Denn im Gegensatz zu 9-Volt-Batterien gibt es keine Akkus für Knopfzellen.
Beruhigend ist die geringe Leistungsaufnahme, was größte Argument für die genutzte Hard- und Software 
war.

\chapter{Bluetooth-Low-Energy}
Mit Bluetooth 5.0 wurde eine neue "Ubertragunsweise zu 
Bluetooth hinzugefügt. Diese nennt sich Bluetooth-Low-Energy und zeichnet
sich durch einen geringen Stromverbrauch und einem höheren 
Datendurchsatz aus. \\
\\
Bluetooth sendet Daten in Paketen. Hierbei ist bei Bluetooth-Low-Energy
(zukünftig BLE) der Sender als Server ausgewiesen und der Empfänger als Client.\\
\\
Die Server bieten \textit{Services} an, die mit \textit{Characteristics} befüllt sind.
So bietet mein Arduino den Service \textit{MPU9250} an, mit dem Characteristis \textit{Accl}, \textit{Gyro}
und \textit{Mag}.\\
Der Nachteil dieser Verteilung der einzelnen Daten besteht hierbei in der Zeit, die für die
Abfrage gebraucht wird. Jede \textit{Characteristic} muss einzeln abgefragt werden, hierbei kann ein
Großteil des Datendurchsatzes des MPU9250 verloren gehen.\\
\\
Laut Dokumentation beträgt der maximale Datensatz von BLE 244 Bytes pro Paket bei 
aktiviertem DLE. Diese Funktion ließ ich ausgeschaltet, wodurch ich maximal
27 Bytes pro Paket versenden kann. Dieses Problem erklärt ebenfalls weshalb die Sensor-Daten
auf verschiedene \textit{Characteristics} aufgeteilt werden. Alle Daten passen nicht in ein einzelnes
zu versendendes Paket. \\
\\

\section{Datengröße}
Die Daten werden als String versendet, diese werden von Arduino mit einer
Null terminiert. \\
\\
Die Größe der Sensordaten beträgt:\\
\textit{Vorkommastellen (3) + Komma (1) + Dezimalstellen (2) + Terminierung (1) = 7 Char}\\
\\
1 Char entspricht 1 Byte, somit gilt:\\
\textit{
9 Sensoren * 7 Byte = 63 Byte \\
63 Byte / 27 Byte = 2,3 Datenpakete pro alle Sensoren
}\\
\\
Somit brauche ich für das Senden aller Sensoren mindestens drei \textit{Characteristics}.


\section{Datendurchsatz per BLE}
%Quelle:
%https://www.novelbits.io/bluetooth-5-speed-maximum-throughput/
%--------------------------------------------------------------

Das Sendeprotokoll von Bluetooth schreibt vor, dass ein Datenpaket von
leeren Datenpaketen eingepackt wird. Ebenso ist eine kurze Wartezeit vorgeschrieben. 
Diese beträgt 150 Mikrosekunden und wird abgekürzt mit \textit{IFS}.
Der Arduino Nano unterstützt \textit{2 Mbps} bei der BLE-Übertragung, dies ist also die Datenrate.
Des Weiteren wird nicht auf eine Antwort des \textit{Clients} gewartet, was die 
Übertragungsgeschwindigkeit weiter erh"oht.\\
\\
Somit beträgt die optimale Sendezeit pro Datenpaket:\\
\textit{Zeit = Sendedauer[Leer] + IFS + Sendedauer[Voll] + IFS\\
Sendedauer[Leer] = LeeresPacket / Datarate}\\
\\
Für mich heißt das:\\
\textit{LeeresPacket = 2 + 4 + 2 + 3 = 11 Bytes = 88 bit}\\
\\
und die Sendezeit für das leere Paket beträgt damit:\\
\textit{Sendedauer[Leer] = 88 bit / 2 Mbps = 44 Mikrosekunden}\\
\\
Für ein volles Datenpaket brauche ich:\\
\textit{2+4+2+4+27+3 = 42 Byte (* 8 Umrechnung in Bit)  = 336 bit\\
Sendedauer[Voll] = 336 bit / 2 Mbps = 168 Mikrosekunden}\\
\\
Für ein gesamtes Datenpaket brauche ich somit mindestens:\\
\textit{Zeit = 44 + 150 + 168 + 150 = 512 Mikrosekunden}\\
\\
beziehungsweise 0,512 Millisekunden. Die maximal erreichbare Datenübetragungs-Frequenz
liegt bei 1,95 kHz. 

%-----------------------------------------------------------------

\section{Tatsächliche Übetragunsgeschwindigkeit}
Die ausgerechnete Datenrate kann in der Praxis kaum erreicht werden, weshalb ein Test zur 
tats"achlichen Datenrate Pflicht ist. Für den Test wurde der auch später in der Praxis verwendete
Code verwendet. Die gemessene Datenrate entspricht 50 Hz, was der eingestellten Aktualisierungsrate 
des MPU9250 entspricht. 

\section{Analyse}
Da der Schussablauf eines Bogenschützen viele Stationen mit verschiedenen Bewegungen 
beinhaltet, fällt es häufig sogar den Trainern schwer, zwischen einem technisch guten
oder schlechten Schuss zu unterscheiden.\\ 
Somit galt es einen Punkt zu finden, bei dem Fehler auffällig sind, so dass ich die Daten auch in 
der Praxis nachvollziehen kann.\\
\\
Eine interessante Bewegung stellt vor allem der Zugarm des 
Schützen dar. Der Auszug verläuft nahezu linear, der häufigere Fehler 
an dieser Stelle versteckt sich in der Höhe des Zugarms. 
Um diese zu messen, muss man die Erdanziehungskraft der Z-Achse herausrechnen.\\
Möglich ist ebenso, die Neigung des Armes über die Euler-Winkel zu berechnen. Sollte 
der Arm zu hoch sein, wird sich die Ausrichtung des Sensors stark verändern, so wie 
die zurückgelegte Distanz in Z-Richtung.\\
\\
Um den vollen Bewegungsablauf eines Schützen zu tracken, wird an jedem Körperteil ein Sensor
benötigt. Dies ermöglicht nicht nur das gegenseitige Überprüfen auf Sinnhaftigkeit der Orientierung
(menschliches Skelett hat beschränkte Bewegungsfreiräume), sondern auch die Klassifizierung der 
einzelnen Schuss-Abläufe könnte, auf Grund von mehr Daten, genauer ablaufen.\\ 
\\
Der MPU9250 bietet 9 Freiheitsgrade (Degrees Of Freedom - DOF). Aus welchen diese bestehen, wird
im Kapitel zum MPU9250 erläutert. Diese Freiheitsgrade lassen es zu, die Distanz
die der Sensor sich in alle Richtungen bewegt, zumindest in der Theorie, zu berechnen. Die Orientierung des Sensors 
und damit die Orientierung des Körpers an der er befestigt ist, kann man genau messen und berechnen. \\
\\
So ist eine Darstellung als 3D-Modell möglich, an der der Schütze seine Fehler sieht und Unterschiede
zu vorangegangen Schüssen hervorgehoben werden. \\
Denkbar wäre eine Klassifizierung der einzelnen Schüsse, um, wie in einem Videospiel, die Genauigkeit
der Wiederholung in Prozent anzugeben. Interessant wird es, sobald mehrere Schützen Datensätze
ihrer Schüsse vergleichen. Unterschiede werden ersichtlich, genau wie Gemeinsamkeiten. \\
\\

Dieses Kapitel beschäftigt sich im Folgenden mit der dreidimensionalen Visualisierung.

\section{Integrale}
Die Distanz, berechenbar aus den Werten des Beschleunigungssensors, kann uns die Differenz zwischen einem Startpunkt
und dem jetzigen Punkt des Sensors berechnen. Durch die drei Achsen meines Beschleunigungssensors ginge dies sogar in alle Richtungen. Damit 
könnte man ein Modell erzeugen, das den betreffenden Punkt in die einzelnen Richtungen verschiebt, bis der finale
neue Standort des Sensors erreicht wird. \\ 
\\ 
Um die Distanz aus der gemessenen Beschleunigung zu berechnen,
fand ich zwei verschiedene Formeln. Die wohl bekannteste Umrechnung 
benutzt Integrale, die zweite Formel ist die der gleichm"aßigen 
Beschleunigung.\\
\\
Die Integration wird von allen mir bekannten Forschungen verwendet. 
Man muss eine Doppelintegration
ausführen um von Beschleunigung auf Distanz zu kommen, hierbei verwandelt 
sich das Rauschen des Sensors in Drift, so entsteht einen exponentiell steigender
Fehler. \\ 
Die gleichmäßige Formel kann, im Gegensatz zum Integral, nur positive 
Beschleunigungen verwerten, hat in den Tests allerdings 
deutlich genauere Werte und einen geringeren Fehler bei Stillstand 
aufgezeigt. \\
\\
Als Zeit wird die Frequenz, mit der der Sensor Daten misst, 
genommen. Hierfür wird die Frequenz in Zeitabschnitte umgerechnet, mit 
der die Formeln letzendlich arbeiten. Der Testaufbau und die Durchführung 
ist im \textit{ 7.1 Distanz-Testaufbau} zu sehen. Die Formel entspricht hierbei der 
Umrechnung in mHz, da ein Skalierungsfehler bei der Verwendung der normalen \textit{1/f}-Formel entsteht.
\subsection{Gleichmäßige Formel}
Die angepasste Formel f"ur gleichm"aßige Beschleunigung berechnete
die Distanz in meinen Versuchen mit einer Genauigkeit von
$\pm8$ cm auf 30cm Teststrecke.\\
Die physikalische Formel für Bewegungen mit einer Anfangsgeschwindigkeit lautet 
$s + v * t  + a * t^2 * 0.5$,
allerdings wurden mit dem leicht verändertem Weg-Zeit-Gesetz $s + a * t^2 * 0.5$ 
bessere Testergebnisse erzielt.\\
Da die negative Beschleunigung den Wert der Distanz wieder nullierte,
wurden zu dieser Formel nur positive Distanzmessungen zugelassen.\\
Hier der Code-Ausschnitt:
\begin{verbatim}
    if (acc > 0) {
    t = (freq / 1000); //hz is not time but frequenzy

    distance = (distance) + (acc * (t * t) * 0.5);
    velocity = acc * t;
    }
\end{verbatim}
F"ur die Tests wurde dieser Code statt der BLE-"Ubetragung auf dem Arduino ausgef"uhrt. 

\subsection{Integral}
Die Berechnung der Distanz über Integrale ist der Standard 
in der Wissenschaft. Beim integrieren meiner Sensordaten ist zu beachten, 
das beim mathematischen Integrieren die zu berechnenden Werte gegen den Limes 
laufen können. Da ich nur begrenzete Datenmengen zur Verfügung habe, ist dies 
nicht möglich, wodurch allein das Integrieren Fehler erzeugt. Ebenfalls wird das 
Rauschen meines Sensors nicht beachtet und aufaddiert. Da ich zwei mal integrieren 
muss, ist dieser Fehler in der Distanz exponentiell.\\
\\
Die Testergebnisse ergaben einen Fehler von $\pm15$cm auf einer Strecke
von 30cm. Ebenso war das Ergebnis sofort in Zentimetern, statt wie erwartet in Metern. 
Wurde der Sensor zurückbewegt an seinen Startpunkt, sank
der Wert jedoch wieder auf Null ab. Somit kann man schließen,
dass das Ergebniss nur falsch skaliert ist.\\
Dieser Fehler wurde noch nicht behoben. \\
Ein klarer Vorteil dieser Rechnung zeigt sich schon beim Test,
die Formel funktioniert auch für negative Beschleunigungen.
Der Wert sinkt am Ende der Teststrecke nicht ab, sondern bleibt
auf seinem hohen Wert.\\
Folgend der Code-Ausschnitt der Integral-Rechnung:
\begin{verbatim}
    t = (freq / 1000); // hz to time

    velocity = t * ((acc + accOld) / 2) + velocity;
    accOld = acc;
    distance = t * ((velocity + velocityOld) / 2) + distance;
    velocityOld = velocity;  
\end{verbatim}

\section{Orientierungen}
Um die Orientierung des Sensors zu berechnen, geht man auf die Grundaussagen der Sensoren zurück. Der 
Beschleunigungssensor gibt die G-Kräfte an, daraus lassen sich Rückschlüsse über die Orientierung 
der Pitch- und Roll-Achse ziehen. Das Gyroskop gibt Drehmoment-Werte zurück, worüber sich der Beschleunigungssensor
überprüfen lässt und eine Drehung um die Yaw-Achse messbar macht. Das Magnetometer setzt schließlich alle Werte in
Relation zum Erdmagnetfeld und kann sowohl das Gyroskop als auch den Beschleunigungssensor überprüfen. Diese Überprüfung
findet über Filter statt, in meinem Projekt verwende ich hierfür den AHRS-Filter\footnote{Der AHRS-Filter 
ist ein erweiterter Kalman-Filter. Er passt seine Fehlerwerte über Zeit dem Drift der Sensorik an und bietet so auch 
auf lange Zeit gute Ergebnisse.}. Der AHRS-Filter fusioniert die Daten in eine verwendbar Orientierung.\\
Für die Darstellung der Orientierung im dreidimensionalen Raum gibt es mehrere Ansätze, die zwei bekanntesten 
sind wohl die Euler-Winkel und die Quaternionen. Die Vor- und Nachteile werden im Folgenden erläutert.
\subsection{Euler-Winkel}
Bei Euler-Winkeln wird das Objekt um die einzelnen Achsen in einer festgelegten Reihenfolge gedreht. Sollten hierbei 
nach einer Drehung zwei Achsen aufeinander liegen, entsteht der sogenannte \textit{Gimbal-Lock} und die Darstellung
der Bewegung findet fehlerhaft statt. Für reele Systeme, wie Gimbal an einer Drohne, ist dieser Fehler vernachlässigbar,
da er nur selten vorkommt und die Euler-Winkel den tatsächlich benötigten Drehwinkeln der Motoren entsprechen.
Da ich jedoch keine Motoren ansteuern muss, steht vor allem der Fehler im Mittelpunkt und ist der Grund dafür, das ich nicht
die Euler-Winkel verwende.  

\subsection{Quaternionen}
Um die Position eines Körpers im dreidimensionalen Raum darzustellen, reichen drei Koordinaten, 
die Möglichkeit der Orientierung verlangt jedoch nach mindestens einer vierten Koordinate. 
Quaternionen beschreiben ein solches vierdimensionales System.
Die Formel der Quaternionen sieht eine Reelle und drei imaginäre Zahlen vor.
Die imaginären Zahlen im Verhältnis zueinander stehen und ihrere Werte, vereinfach gesagt, skaliert werden können, kann 
ein ''Richtungsvektor'' berechnet werden. Dieser kann in alle Richtungen zeigen. Das Objekt wird an diesem Vektor orientiert.
Ein Überprüfen der Rohdaten fällt durch die vierte Dimension für mich aus.
Allerdings schaffen es die Quaternionen durch diese weitere Dimension jegliche Orientierung im Raum darzustellen ohne
etwas ähnliches wie einen \textit{Gimbal-Lock} zu erzeugen. Die Vollständige Erklärung der Quaternionen sprengt momentan
leider meinen Wissenstand. Die einfache Eingabe meiner Daten in fast alle bekannte 3D-Visualisierungsprogramme und 
der fehlende \textit{Gimbal-Lock} sind der Grund für mich, zukünftig die Quaternionen zu verwenden.

\chapter{Software}
Die Anzeige für Daten erfolgt am Handy, hier habe ich genug Rechenpower
nicht nur Nachkommastellen genau zu berechnen, sondern kann ebenfalls 
visuelle Darstellungen erzeugen. Die Daten werden hierfür über das bereits
erklärte BLE an das Handy gesendet.
Somit brauchte ich sowohl eine Android App als auch ein Programm für den Arduino.
\section{Datenübertragung}
Die Daten werden wie in Kapitel zur BLE-Datenübetragung beschrieben in maximal
3 verschiedenen Charakteristiken gesendet und empfangen. \\
Eine Instanz muss hierbei sicherstellen das Daten
nicht doppelt gesendet oder empfangen werden.\\
Dies wird beim Arduino durch die Abfrage einer bibliothekseigenen Funktion
sichergestellt, die erst auslöst wenn neue Daten des MPU9250 erzeugt wurden. Dies bewirkt
das Bereitstellen der Daten in den Characteristics.\\
Sobald diese Daten gesendet wurden, bekommt das Handy ein Signal und liest
daraufhin die neuen Daten.

\section{Arduino Firmware}
Das Programm auf dem Arduino Nano 33 Ble stellt zu Beginn eine 
I$^2$C Verbindung mit dem MPU9250 her und startet das BLE-Modul.\\
Sobald ein Ger"at sich mit dem Arduino verbindet beginnt dieser
mit der Datenübertragung.\\
Hierbei wurden die Filter vom Digital-Motion-Processors des MPU9250 
schon mit einem Low-Pass-Filter verarbeitet. 
\section{Android-App}
Zur Erstellung der Android-App wurde AppInventor und die BLE-Extension verwendet.\\
\\
Beim Start der App wird man aufgefordert Bluetooth und GPS anzuschalten, das GPS
ist nach Android-Richtlinien zu aktivieren. Danach kann man nach verschiedenen
Geräten scannen und sich mit diesen zu verbinden. Erfolgt die Verbindung mit einem
falschen Gerät schließt sich die App.\\
Nach der Verbindung wird sofort die Übertragung gestartet\\
\\
Das empfangene Datenpaket muss vor der Verarbeitung in die einzelnen Daten
aufgespalten und von String zu mindestens Float-Werten gepaarst werden.
Die Split-Funktion von AppInventor sucht nach ``|'' als Trennzeichen und spaltet
hier die Werte. Diese werden in ein Array gespeichert welches später
die einzelnen aktuellen Werte ausgeben kann.\\
Es gibt insgesamt 3 Möglichkeiten die Daten anzeigen zu lassen.
%\begin{itemize}
%    \item Rohdaten (Live)
%    \item Graph (Live)
%    \item Aufgezeichnete Daten
%\end{itemize}

\subsection{Rohdaten}
Die gelesenen Daten werden direkt im Textformat auf dem Bildschirm ausgegeben.
Der Zeitunterschied zwischen Datenpaketen wird in Millisekunden auf dem Bildschirm
angezeigt.
Es ist möglich die Daten gleichzeitig aufzuzeichnen.

\subsection{Graph}
Die Daten werden in einem Graph dargestellt, hierzu werden sie zuerst in ein
Array geschrieben. Aus diesem Array erzeugt das Programm in einem vorgegebene Bereich
die Datenpunkte, die aufgrund ihrer Masse wie ein Liniendiagramm aussehen.\\
Neue Daten werden rechts geschrieben, während die alten Daten nach Links aus dem
Bildschirm verschwinden.\\
Der Zeitunterschied zwischen Datenpaketen wird in Millisekunden auf dem Bildschirm
angezeigt.
Es ist möglich die Daten gleichzeitig aufzuzeichnen.

\subsection{Aufgezeichnete Daten}
Die von den anderen Funktionen aufgezeichneten Funktionen können hier ausgegeben
werden. Hierzu benötigt der Schütze den Namen der Datei. Die Dateien werden
seit kurzem unter Android in einem App-Eigenem Ordner gespeichert. Diesen muss
der Schütze momentan auslesen um den zufälligen Namen der neuen Datei zu kennen.\\
Die Daten werden als Graph dargestellt. Die Beschleunigungsdaten werden außerdem
in Rohform über dem Graphen ausgegeben.
\section{Gleichmäßige und ungleichmäßige Beschleunigung}
Um die Distanz aus der gemessenen Beschleunigung zu berechnen,
fand ich zwei verschiedene Formeln. Die wohl bekannteste Umrechnung 
benutzt Integrale, die zweite Formel ist die der gleichm"aßigen 
Beschleunigung.\\
\\
Die Integration wird von allen mir bekannten Forschungen verwendet. 
Man muss eine Doppelintegration
ausführen um von Beschleunigung auf Distanz zu kommen, hierbei verwandelt 
sich das Rauschen des Sensors in Drift und so einen exponentiell steigenden
Fehler. \\ 
Die gleichmäßige Formel kann im Gegensatz zum Integral, nur positive 
Beschleunigungen verwerten, hat in den folgenden Tests allerdings 
deutlich genauere Werte und einen geringeren Fehler bei Stillstand 
aufgezeigt. So wird in diesem Projekt die gleichmäßige Beschleunigungsformel 
verwendet.\\
\\
Als Zeit wird die Frequenz, mit der der Sensor Daten misst, 
genommen. Hierfür wird die Frequenz in Zeitabschnitte umgerechnet, mit 
der die Formeln letzendlich arbeiten.
\subsection{Gleichmäßige Formel}
Die Formel für gleichmäßige Beschleunigung berechnete 
die Distanz in meinen Versuchen mit einer Genauigkeit von
+-10 cm auf 30cm Teststrecke. Wurden hierbei ebenfalls 
negative Beschleunigungen gemessen wirkten sich diese direkt
auf die Distanz aus. Dies stellte ein Problem beim abbremsen
am Ende der Teststrecke dar. Die Werte sanken wieder auf null.\\
Aus diesem Grund sind nur positive Werte für diese Funktion 
zugelassen. Hier der Code-Aussschnitt aus der Berechnung.\\
Für die Tests wurde dieser Code statt der BLE-Übetragung 
auf dem Arduino ausgeführt.

\begin{verbatim}
    if (acc > 0) {
    t = (freq / 1000); //hz is not time but frequenzy

    distance = (distance) /*+ (velocity * t) */ + (acc * (t * t) * 0.5);
    velocity = acc * t;
    }
\end{verbatim}

\subsection{Integral}
Die Berechnung der Distanz über Integrale ist ``The way to go''
in der Wissenschaft, obwohl die Fehler die sie mit sich bringt
bekannt sind. Eben diese Fehler wirkten sich bei meinem Sensor
stark aus.\\
Die Testergebnisse ergaben einen Fehler von +70cm auf einer Strecke
von 30cm. Wurde der Sensor zurückbewegt an seinen Startpunkt sank
der Wert jedoch wieder auf Null ab. Somit kann man schließen,
dass das Ergebniss nur falsch skaliert ist.\\
Dieser Fehler wurde noch nicht behoben. \\
Ein klarer Vorteil dieser Rechnung zeigt sich schon beim Test,
die Formel funktioniert auch für negative Beschleunigungen.
Der Wert sinkt am Ende der Teststrecke nicht auf, sondern bleibt
auf seinem hohen Wert.\\
Folgend der Code-Ausschnitt der Integral-Rechnung:
\begin{verbatim}
    t = (freq / 1000); // hz to time

    velocity = t * ((acc + accOld) / 2) + velocity;
    accOld = acc;
    distance = t * ((velocity + velocityOld) / 2) + distance;
    velocityOld = velocity;  
\end{verbatim}
\section{Tests}
\subsection{Distanz-Messungen}
Der Sensor wurde auf einem Breadboard mit einem Arduino Uno verbunden und übertrug die Werte
vie USB-Kabel an den Seriellen Monitor der Arduino IDE. Um eine mögliche Schieflage auf dem
Breadboard abzufedern wurden aus 200 Messungen der Durchschnitt berechnet und von den Werten vor
der Verarbeitung subtrahiert. So liegt der Sensor mathematisch absolut flach auf. \\
Insgesamt wurden 10 Tests pro Formel gemacht, die Ergebnisse können Sie unten einsehen.\\

\begin{figure} [h]
    \centering
    \includegraphics[width = 15cm]{Bilder/_DistanzVergleich}
    \caption{Distanzen | Alle Tests}
    %\label{fig:wo-bin-ich}
    \end{figure}

\subsection{Fehler}
Die zwei weiteren Graphen zeigen die Fehler bei 10 Sek"undigem Stillstand. Der exponentielle Drift 
bei der Integralen Formel ist typisch.

\begin{figure} [h]
    \centering
    \includegraphics[width = 15cm]{Bilder/_integralDistance001}
    \caption{Integrale Formel | Drift}
    %\label{fig:wo-bin-ich}
    \end{figure}

    \begin{figure} [h]
        \centering
        \includegraphics[width = 15cm]{Bilder/_constDistance001}
        \caption{Gleichf"ormige Beschl. Formel | Drift}
        %\label{fig:wo-bin-ich}
        \end{figure}
    


%\chapter{Machine Learning}
%\section{Datenauswahl und Dimensionen}
%\input{MLData}
%\section{Modell und Funktion}
%\input{MLModNFunk}

\chapter{Fazit}
Die Distanzmessung über den MPU9250 funktioniert 
mit einem relativ großem Fehler, der es schwer macht
die Daten direkt in ein 3D-Programm einzufügen. 
Danke an \\
\begin{itemize}
    \item Herr Czernohous (Korrekturlesen der Arbeit und technische Hilfe, Kauf der Materialien)
    \item Herr Dierle (Hilfe bei physikalischen Zusammenhängen und Formeln, Korrekturlesen der Arbeit)
    \item Albert Stubbe (Wenn man den Baum vor lauter Wäldern nicht sieht)
\end{itemize}
\bibliography{bibliography}
\bibliographystyle{plain}
\end{document}
