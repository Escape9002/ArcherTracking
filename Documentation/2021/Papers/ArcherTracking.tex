\documentclass[12pt,titlepage]{scrreprt}
\usepackage[ngerman]{babel}
\usepackage[utf8]{inputenc}
\usepackage{color}
\usepackage{float}
\usepackage[a4paper,lmargin={2.5cm},rmargin={2.5cm},tmargin={2.5cm},bmargin = {2cm},footskip={1cm}]{geometry}
\usepackage{amssymb}
\usepackage{amsthm}
\usepackage{graphicx}
\usepackage{subfig}
\usepackage{wrapfig}
\usepackage{url}
\usepackage{calc}
\usepackage{overcite} 
\renewcommand\citeform[1]{[#1]}
\usepackage{hyperref}
\hypersetup{
% colorlinks=false,
  pdfborder={0 0 0}
}

% !TEX program = xelatex
\begin{document}
% \include{Title}
\begin{titlepage}

	

\title{Archer-Tracking}
\subtitle{Bewegungstracking für Sportler}
\titlehead{\centering\includegraphics[width=15cm]{Bilder/IMG_9973 (4)}}


\author{Antonio Rehwinkel}

\publishers{Unterstützende Lehrer: Herr Czernohous \\ Herr Dierle \\  \texttt{Emails?} \\
\vspace*{2ex} und Professoren: Hochschule1 \\ Hochschule2\\ \texttt{Emails?}}
%- \\ Schiller-Gymnasium Offenburg}

\maketitle

\end{titlepage}
\tableofcontents
\chapter[Vorwort]{Vorwort}

Mithilfe eines Beschleunigungssensoren kann man viele 
Bewegungen erforschen und vermessen.
Mit meinem System sollen mehrere Beschleunigungssensoren 
dazu eingesetzt werden können, Bewegungsabläufe aufzunehmen, 
miteinander zu vergleichen und zu erkennen.\\
\\
Die Daten werden über Bluetooth-Low-Energy an ein Handy 
geschickt, wo Sie sowohl gespeichert als auch verwertet 
werden können. Als Beispiel gilt hier für mich das Bogenschießen, 
bei dem selbst kleine Bewegungen immer wieder auf gleiche Weise 
ausgeführt werden müssen. Mit meinen Sensor sollen hier teure 
Kamerasysteme abgeschafft werden und es so jeden ermöglichen, 
selbst ohne Bogen oder Trainer bei sich Zuhause zu den 
Bewegungsablauf zu trainieren.\\

\section{IMU vs Kameratracking}
Mein Projekt überschneidet sich in seinen Zielen häufig 
mit Tracking das bei VR-Brillen eingesetzt wird. Hier wird
zur Feststellung der Position des Spielers häufig eine 
Kombination aus Kameratracking und Infrarot-LED. \\
Hierbei muss der Spieler die Fernbedienungen festhalten die
die Infrarot-LEDs beinhalten, während die Kameras im Raum 
so verteilt werden müssen das der Spieler immer erkannt wird.\\
Die Neigung des Kopfes und der Hände werden auch hier häufig 
Mithilfe eines IMU bestimmt.\\
\\
Da diese Systeme viel Platz benötigen, viel Geld kosten und 
für die Bildverarbeitung häufig eine große Rechenkraft benötigen
ist dieses System nicht für viele Privatnutzer sinnvoll oder 
bieten einen Bewegungsfreiraum der Sport zu lässt.\\


%\\
%Die IMU-Sensoren bestehen mindestens aus einem Gyroskop und einem 
%Beschleunigungssensor, manche bieten sogar ein Magnetometer an.
%Somit sollte es möglich sein, über die
%Beschleunigung die Distantz die ein Körper mit diesem Sensor zurück
%legt zu messen. Die Neigung und Orientation sind über das Gyroskop 
%und Magnetometer sehr genau messbar.\\
%\\
%Die Vorteile der IMU liegen auf der Hand, sie sind günstig, klein
%und leicht. Aus diesen Gründen trägt fast jeder heutzutage 
%so einen Sensor bei sich, die meisten Handys haben ihn schon eingebaut.\\
%\\
%Für mein Projekt benutze ich dennoch einen eigenen MPU, um die
%Qualität der Messdaten sicher zu stellen.


\section{Bewegungen und Analyse}
Eine Interessante Bewegung stellt vor allem der Zugarm des 
Schüzten dar. Der Auszug verläuft nahezu linear, der häufigere Fehler 
an dieser Stelle versteckt sich allerdings in der Höhe des Zugarms. 
Um diese zu messen muss man die Erdanziehungskraft der Z-Achse herausrechnen.\\
\\
Da der Schussablauf eines Bogenschützen viele Stationen mit verschiedenen Bewegungen 
beinhaltet fällt es häufig sogar den Trainern schwer zwischen einem technisch guten
oder schlechtem Schuss zu unterscheiden.\\
Die Datenlage aus dem 9DOF-System des verwendeten MPU erzeugt eine Datenwolke, die 
diesen Vorgang fürs erste verkompliziert.\\
Die Daten sind allerdings sehr gut zu vergleichen und zu mitteln. Mit diesen 
Eigenschaften kann man über künstliche Intelligenz, genauer, Machine Learning 
die Schüsse klassifiezieren. So kann jeder Schütze seine eigene Datenbasis erstellen,
nach der sein Schuss klassifiziert und so eingeordnet werden können.\\
Ein Vergleich mit einer deutlich größeren Datenbasis als einem einzelnen Schützen ist denkbar.\\
\\
Für dieses System ist es nahezu unwichtig wie viele Körperteile überwacht werden.
Mit mehr Sensoren (oder Vergleichspunkten) wird es lediglich schwieriger für den
Schützen einen guten Wert bei der klassifiezierung zu erreichen.\\
Die Verwendung setzt natürlich vor allem korrekte, genau und viele Daten vorraus,
dies gilt für das Training des Modells so wie für die Live-Daten des Schützen.\\
\\


\chapter{IMU vs Kameratracking}
Mein Projekt überschneidet sich in seinen Zielen häufig 
mit Tracking das bei VR-Brillen eingesetzt wird. Hier wird
zur Feststellung der Position des Spielers häufig eine 
Kombination aus Kameratracking und Infrarot-LED. \\
Hierbei muss der Spieler die Fernbedienungen festhalten die
die Infrarot-LEDs beinhalten, während die Kameras im Raum 
so verteilt werden müssen das der Spieler immer erkannt wird.\\
Die Neigung des Kopfes und der Hände werden auch hier häufig 
Mithilfe eines IMU bestimmt.\\
\\
Da diese Systeme viel Platz benötigen, viel Geld kosten und 
für die Bildverarbeitung häufig eine große Rechenkraft benötigen
ist dieses System nicht für viele Privatnutzer sinnvoll oder 
bieten einen Bewegungsfreiraum der Sport zu lässt.\\


%\\
%Die IMU-Sensoren bestehen mindestens aus einem Gyroskop und einem 
%Beschleunigungssensor, manche bieten sogar ein Magnetometer an.
%Somit sollte es möglich sein, über die
%Beschleunigung die Distantz die ein Körper mit diesem Sensor zurück
%legt zu messen. Die Neigung und Orientation sind über das Gyroskop 
%und Magnetometer sehr genau messbar.\\
%\\
%Die Vorteile der IMU liegen auf der Hand, sie sind günstig, klein
%und leicht. Aus diesen Gründen trägt fast jeder heutzutage 
%so einen Sensor bei sich, die meisten Handys haben ihn schon eingebaut.\\
%\\
%Für mein Projekt benutze ich dennoch einen eigenen MPU, um die
%Qualität der Messdaten sicher zu stellen.


\chapter{Bluetooth-Low-Energy}
Mit Bluetooth 5.0 wurde eine neue "Ubertragunsweise zu 
Bluetooth hinzugefügt. Diese nennt sich Bluetooth-Low-Energy und zeichnet
sich durch einen geringen Stromverbrauch und einem höheren 
Datendurchsatz aus. \\
\\
Bluetooth sendet Daten in Paketen. Hierbei ist bei Bluetooth-Low-Energy
(zukünftig BLE) der Sender als Server ausgewiesen und der Empfänger als Client.\\
\\
Die Server bieten \textit{Services} an, die mit \textit{Characteristics} befüllt sind.
So bietet mein Arduino den Service \textit{MPU9250} an, mit dem Characteristis \textit{Accl}, \textit{Gyro}
und \textit{Mag}.\\
Der Nachteil dieser Verteilung der einzelnen Daten besteht hierbei in der Zeit, die für die
Abfrage gebraucht wird. Jede \textit{Characteristic} muss einzeln abgefragt werden, hierbei kann ein
Großteil des Datendurchsatzes des MPU9250 verloren gehen.\\
\\
Laut Dokumentation beträgt der maximale Datensatz von BLE 244 Bytes pro Paket bei 
aktiviertem DLE. Diese Funktion ließ ich ausgeschaltet, wodurch ich maximal
27 Bytes pro Paket versenden kann. Dieses Problem erklärt ebenfalls weshalb die Sensor-Daten
auf verschiedene \textit{Characteristics} aufgeteilt werden. Alle Daten passen nicht in ein einzelnes
zu versendendes Paket. \\
\\

\section{Datengröße}
Die Daten werden als String versendet, diese werden von Arduino mit einer
Null terminiert. \\
\\
Die Größe der Sensordaten beträgt:\\
\textit{Vorkommastellen (3) + Komma (1) + Dezimalstellen (2) + Terminierung (1) = 7 Char}\\
\\
1 Char entspricht 1 Byte, somit gilt:\\
\textit{
9 Sensoren * 7 Byte = 63 Byte \\
63 Byte / 27 Byte = 2,3 Datenpakete pro alle Sensoren
}\\
\\
Somit brauche ich für das Senden aller Sensoren mindestens drei \textit{Characteristics}.


\section{Datendurchsatz per BLE}
%Quelle:
%https://www.novelbits.io/bluetooth-5-speed-maximum-throughput/
%--------------------------------------------------------------

Das Sendeprotokoll von Bluetooth schreibt vor, dass ein Datenpaket von
leeren Datenpaketen eingepackt wird. Ebenso ist eine kurze Wartezeit vorgeschrieben. 
Diese beträgt 150 Mikrosekunden und wird abgekürzt mit \textit{IFS}.
Der Arduino Nano unterstützt \textit{2 Mbps} bei der BLE-Übertragung, dies ist also die Datenrate.
Des Weiteren wird nicht auf eine Antwort des \textit{Clients} gewartet, was die 
Übertragungsgeschwindigkeit weiter erh"oht.\\
\\
Somit beträgt die optimale Sendezeit pro Datenpaket:\\
\textit{Zeit = Sendedauer[Leer] + IFS + Sendedauer[Voll] + IFS\\
Sendedauer[Leer] = LeeresPacket / Datarate}\\
\\
Für mich heißt das:\\
\textit{LeeresPacket = 2 + 4 + 2 + 3 = 11 Bytes = 88 bit}\\
\\
und die Sendezeit für das leere Paket beträgt damit:\\
\textit{Sendedauer[Leer] = 88 bit / 2 Mbps = 44 Mikrosekunden}\\
\\
Für ein volles Datenpaket brauche ich:\\
\textit{2+4+2+4+27+3 = 42 Byte (* 8 Umrechnung in Bit)  = 336 bit\\
Sendedauer[Voll] = 336 bit / 2 Mbps = 168 Mikrosekunden}\\
\\
Für ein gesamtes Datenpaket brauche ich somit mindestens:\\
\textit{Zeit = 44 + 150 + 168 + 150 = 512 Mikrosekunden}\\
\\
beziehungsweise 0,512 Millisekunden. Die maximal erreichbare Datenübetragungs-Frequenz
liegt bei 1,95 kHz. 

%-----------------------------------------------------------------

\section{Tatsächliche Übetragunsgeschwindigkeit}
Die ausgerechnete Datenrate kann in der Praxis kaum erreicht werden, weshalb ein Test zur 
tats"achlichen Datenrate Pflicht ist. Für den Test wurde der auch später in der Praxis verwendete
Code verwendet. Die gemessene Datenrate entspricht 50 Hz, was der eingestellten Aktualisierungsrate 
des MPU9250 entspricht. 

\chapter{Hardware}
\section{Arduino Nano 33 BLE}
Der Arduino Nano 33 BLE wurde mit einem M4-ARM-Processor und einem 5.0 
Bluetooth Modul ausgestattet. Damit ist er BLE-Fähig und liefert einen starken
Prozessor für Kommarechnungen. \\
%Tabelle der Daten einfügen
\\
Der Arduino eignet sich durch seine BLE-Fähigkeit und I2C beziehungsweise
SPI-Anschlüsse zur Verwendung mit dem Verwendetem MPU9250. Die geringe Größe
und Stromverbrauch sind weitere Pluspunkte.

\section{MPU9250}
Der benutze IMU in diesem Projekt ist ein Multi-Chip mit einem
3-Achsen Gyroskop, 3-Achsen Beschleunigungssensor und einem 3-Achsen Magnetometer.
Alle Sensoren wurden noch in der Firma kalibriert. Der Chip bietet einen
eingebauten Low-Pass-Filter der eine erste Bearbeitung der Daten vornimmt 
und so den Hauptprozessor entlastet.\\
In der folgenden Tablle ist die Empfindlichkeit und Genauigkeit der 
einzelnen Sensoren notiert, sowie die Übetragunsart und Geschwindigkeit.\\
%Tabelle
\\
Für eine einfache Handhabung benutze ich die I2C Verbindung zwischen 
Arduino und MPU9250.

\section{Stromverbrauch}
Da dieses Jahr fest steht, welche Daten gesendet werden können, lohnt es 
sich nun auch, die benötigte Leistung zu ermitteln.
Diese wurde mit einem Multimeter am Batterieanschluss in
verschiedenen Modi gemessen. Die Ergebnisse stehen in der Tabelle:\\
%Tabelle: Werte: 
%200mA Auslösung:
%BLE aus: 5
%BLE an: 10.2
%BLE Verbunden: 13
%MPU9250 an: 5
%Bereich: 200 | Auflösung : 0,1 uA | +-1% des Messwerts +- 2 Ziffern
%Quelle:https://github.com/Escape9002/ArcherTracking/issues/5
\\
\begin{tabularx}{0.8\textwidth}{l|X|XX}
    Modul & An/Aus & Verbrauch(in mA) \\
    \hline
    BLE & aus & 5 \\
    \hline
    BLE & an & 10.2 \\
    \hline
    BLE & an und verbunden & 13 \\
    \hline
    MPU9250 & an & 5 \\
\end{tabularx}\\
\\
Der Stromverbrauch lässt sich so berechnen und erleichtert die korrekte Batterie-Wahl.
Die verwendeten Formeln: 
\begin{equation}
    $$
    P = U * I \\
    Für die Einheiten gilt nach Multiplikation mit der Zeit:\\
    Ah * V = Wh \\
    Wh / W = h $\rightarrow$ (U*I*t) / U*I = t \\
    $$
\end{equation}
So hat der Arduino eine Leistungsaufnahme von:
\begin{equation}
    $$
    0,005 A * 7,4 V = 0,037 W
    $$
\end{equation}
\\
Die verschiedenen Batterien-Typen stehen in der folgenden Tabelle:\\
\\
\begin{tabularx}{0.8\textwidth}{l|X|X|X|XX}
    Typ & Laufzeit(in Stunden) & Gewicht & Cut-Off-Spannung & Differenz\footnote{zwischen Cut-Off-Spannung und angebotener Spannung}\\
    \hline
    9V & 40 & 50g &7.2V & $9-7.2 = 2,8$\\
    \hline
    CR2025 & 12 & 2,5g & 2V & $3-2 = 1$\\
    \hline
    2 * CR2025 & 48,6 & 5g & 2V & $6-2 = 4$\\
\end{tabularx}\\
%Tabelle
%9V : 40h, 0,05 KG, CutoffVol: 7.2V (9-7.2 = 2,8)
% CR2025 : 12h , 0,0025 Kg, CutoffVolt: 2V (3-2 = 1)
% CR2025 * 2 = 48,6 h, 0,005 Kg, CutoffVolt: 2V (6-2 = 4)
\\
Um eine Stromversorgung des Arduinos sicher zu stellen, benötigt man 2 Knopfzellen
des Typs CR2025. Dennoch hat die Knopfzelle CR2025 nicht nur eine bessere 
Laufzeit sondern ebenfalls weniger Gewicht, braucht weniger Platz und liefert eine bessere 
Differenz zwischen Cut-Off-Spannung zu angebotener Spannung.
\\
Gegen das Umrüsten auf die Knopfbatterie spricht einzig der Umweltschutz.
Denn im Gegensatz zu 9-Volt-Batterien gibt es keine Akkus für Knopfzellen.
Beruhigend ist die geringe Leistungsaufnahme, was größte Argument für die genutzte Hard- und Software 
war.

\chapter{Software}
\section{Android-App}
Zur Erstellung der Android-App wurde AppInventor und die BLE-Extension verwendet.\\
\\
Beim Start der App wird man aufgefordert Bluetooth und GPS anzuschalten, das GPS
ist nach Android-Richtlinien zu aktivieren. Danach kann man nach verschiedenen
Geräten scannen und sich mit diesen zu verbinden. Erfolgt die Verbindung mit einem
falschen Gerät schließt sich die App.\\
Nach der Verbindung wird sofort die Übertragung gestartet\\
\\
Das empfangene Datenpaket muss vor der Verarbeitung in die einzelnen Daten
aufgespalten und von String zu mindestens Float-Werten gepaarst werden.
Die Split-Funktion von AppInventor sucht nach ``|'' als Trennzeichen und spaltet
hier die Werte. Diese werden in ein Array gespeichert welches später
die einzelnen aktuellen Werte ausgeben kann.\\
Es gibt insgesamt 3 Möglichkeiten die Daten anzeigen zu lassen.
%\begin{itemize}
%    \item Rohdaten (Live)
%    \item Graph (Live)
%    \item Aufgezeichnete Daten
%\end{itemize}

\subsection{Rohdaten}
Die gelesenen Daten werden direkt im Textformat auf dem Bildschirm ausgegeben.
Der Zeitunterschied zwischen Datenpaketen wird in Millisekunden auf dem Bildschirm
angezeigt.
Es ist möglich die Daten gleichzeitig aufzuzeichnen.

\subsection{Graph}
Die Daten werden in einem Graph dargestellt, hierzu werden sie zuerst in ein
Array geschrieben. Aus diesem Array erzeugt das Programm in einem vorgegebene Bereich
die Datenpunkte, die aufgrund ihrer Masse wie ein Liniendiagramm aussehen.\\
Neue Daten werden rechts geschrieben, während die alten Daten nach Links aus dem
Bildschirm verschwinden.\\
Der Zeitunterschied zwischen Datenpaketen wird in Millisekunden auf dem Bildschirm
angezeigt.
Es ist möglich die Daten gleichzeitig aufzuzeichnen.

\subsection{Aufgezeichnete Daten}
Die von den anderen Funktionen aufgezeichneten Funktionen können hier ausgegeben
werden. Hierzu benötigt der Schütze den Namen der Datei. Die Dateien werden
seit kurzem unter Android in einem App-Eigenem Ordner gespeichert. Diesen muss
der Schütze momentan auslesen um den zufälligen Namen der neuen Datei zu kennen.\\
Die Daten werden als Graph dargestellt. Die Beschleunigungsdaten werden außerdem
in Rohform über dem Graphen ausgegeben.
\section{Arduino Firmware}
Das Programm auf dem Arduino Nano 33 Ble stellt zu Beginn eine 
I$^2$C Verbindung mit dem MPU9250 her und startet das BLE-Modul.\\
Sobald ein Ger"at sich mit dem Arduino verbindet beginnt dieser
mit der Datenübertragung.\\
Hierbei wurden die Filter vom Digital-Motion-Processors des MPU9250 
schon mit einem Low-Pass-Filter verarbeitet. 
\section{Gleichmäsige und ungleichmäßige Beschleunigung}
Um die Distanz aus der gemessenen Beschleunigung zu berechnen,
fand ich zwei verschiedene Formeln. Die wohl bekannteste Umrechnung 
benutzt Integrale, die zweite Formel ist die der gleichm"aßigen 
Beschleunigung.\\
\\
Die Integration wird von allen mir bekannten Forschungen verwendet. 
Man muss eine Doppelintegration
ausführen um von Beschleunigung auf Distanz zu kommen, hierbei verwandelt 
sich das Rauschen des Sensors in Drift und so einen exponentiell steigenden
Fehler. \\ 
Die gleichmäßige Formel kann im Gegensatz zum Integral, nur positive 
Beschleunigungen verwerten, hat in den folgenden Tests allerdings 
deutlich genauere Werte und einen geringeren Fehler bei Stillstand 
aufgezeigt. So wird in diesem Projekt die gleichmäßige Beschleunigungsformel 
verwendet.\\
\\
Als Zeit wird die Frequenz, mit der der Sensor Daten misst, 
genommen. Hierfür wird die Frequenz in Zeitabschnitte umgerechnet, mit 
der die Formeln letzendlich arbeiten.
\subsection{Gleichmäßige Formel}
Die Formel für gleichmäßige Beschleunigung berechnete 
die Distanz in meinen Versuchen mit einer Genauigkeit von
+-10 cm auf 30cm Teststrecke. Wurden hierbei ebenfalls 
negative Beschleunigungen gemessen wirkten sich diese direkt
auf die Distanz aus. Dies stellte ein Problem beim abbremsen
am Ende der Teststrecke dar. Die Werte sanken wieder auf null.\\
Aus diesem Grund sind nur positive Werte für diese Funktion 
zugelassen. Hier der Code-Aussschnitt aus der Berechnung.\\
Für die Tests wurde dieser Code statt der BLE-Übetragung 
auf dem Arduino ausgeführt.

\begin{verbatim}
    if (acc > 0) {
    t = (freq / 1000); //hz is not time but frequenzy

    distance = (distance) /*+ (velocity * t) */ + (acc * (t * t) * 0.5);
    velocity = acc * t;
    }
\end{verbatim}

\subsection{Integral}
Die Berechnung der Distanz über Integrale ist ``The way to go''
in der Wissenschaft, obwohl die Fehler die sie mit sich bringt
bekannt sind. Eben diese Fehler wirkten sich bei meinem Sensor
stark aus.\\
Die Testergebnisse ergaben einen Fehler von +70cm auf einer Strecke
von 30cm. Wurde der Sensor zurückbewegt an seinen Startpunkt sank
der Wert jedoch wieder auf Null ab. Somit kann man schließen,
dass das Ergebniss nur falsch skaliert ist.\\
Dieser Fehler wurde noch nicht behoben. \\
Ein klarer Vorteil dieser Rechnung zeigt sich schon beim Test,
die Formel funktioniert auch für negative Beschleunigungen.
Der Wert sinkt am Ende der Teststrecke nicht auf, sondern bleibt
auf seinem hohen Wert.\\
Folgend der Code-Ausschnitt der Integral-Rechnung:
\begin{verbatim}
    t = (freq / 1000); // hz to time

    velocity = t * ((acc + accOld) / 2) + velocity;
    accOld = acc;
    distance = t * ((velocity + velocityOld) / 2) + distance;
    velocityOld = velocity;  
\end{verbatim}

\chapter{Machine Learning}
\section{Datenauswahl und Dimensionen}
\input{MLData}
\section{Modell und Funktion}
\input{MLModNFunk}

\chapter{Fazit}
Die Distanzmessung über den MPU9250 funktioniert 
mit einem relativ großem Fehler, der es schwer macht
die Daten direkt in ein 3D-Programm einzufügen. 
Danke an \\
\begin{itemize}
    \item Herr Czernohous (Korrekturlesen der Arbeit und technische Hilfe, Kauf der Materialien)
    \item Herr Dierle (Hilfe bei physikalischen Zusammenhängen und Formeln, Korrekturlesen der Arbeit)
    \item Albert Stubbe (Wenn man den Baum vor lauter Wäldern nicht sieht)
\end{itemize}
\bibliography{bibliography}
\bibliographystyle{plain}
\end{document}
