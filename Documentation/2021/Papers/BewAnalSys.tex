\chapter{Bewegungen, Analyse und bekannte Systeme}
\section{Analyse}
Da der Schussablauf eines Bogenschützen viele Stationen mit verschiedenen Bewegungen 
beinhaltet fällt es häufig sogar den Trainern schwer zwischen einem technisch guten
oder schlechtem Schuss zu unterscheiden.\\ 
Somit galt es einen Punkt zu finden bei dem Fehler auffällig sind, so das ich die Daten auch im 
Leben nachvollziehen kann.\\
\\
Der MPU9250 bietet 9 Freiheitsgrade (Degress Of Freedom - DOF). Aus welchen diese bestehen wird
im Kapitel zum MPU9250 erläutert. Diese Freiheitsgrade lassen es zu, die Orientierung des Sensors 
und damit die Orientierung des Körpers an der er befestigt ist genau zu messen. Sogar die Distanz 
die der Sensor sich bewegt ist in der Theorie messbar.\\
\\
So ist eine Darstellung als 3D-Modell möglich an der der Schütze seine Fehler sieht und Unterschiede
zu vorangegangen Schüssen hervorgehoben werden. \\
Denkbar wäre eine Klassifiezierung der einzelnen Schüsse um wie in einem Videospiel die Genauigkeit
der Wiederholung in Prozent anzugeben. Interessant wird es, sobald mehrere Schützen Datensätze
ihrer Schüsse vergleichen. Unterschiede werden ersichtlich, genau wie Gemeinsamkeiten.\\
\\
Die benötigten Daten für ein 3D-Modell varierien je nach gewollter Genauigkeit der Darstellung.
Mithilfe von Euler Winkeln ist es möglich die Orientierung des Sensors zu berechnen und so nur
die gemessene Beschleunigung "ubertragen zu m"ussen. Allerdings verliert die Darstellung so einen
Freiheitsgrad durch den sogennanten ``Gimbal-Lock''. Die Euler Winkel drehen das Objekt um gedachte
Achsen herum, wenn dabei eine Axe um 180° gedreht wird würde es zu einer zweimaligen Drehung
um die selbe Axe kommen. Ebenfalls gestalltet sich die Berechnung der Gier-Axe schwer, da hier
keine Veränderung in eine der gemessenen Axen für Beschleunigung erkennbar ist.
\\
Um dieses Problem zu lösen muss man das Gyroskop mit einbeziehen, dieses kann die Beschleunigung
in Winkeln bemessen. Damit könnte man die Orientierung eines Objekt bestimmen, allerdings ohne 
Bezug zur Orientierung in der echten Welt. Man wüsste nicht wie das Objekt steht, man weiß nur 
das es sich um einen bestimmten Winkel in diese oder die andere Richtung gedreht hat. \\
Deshalb muss man die Daten des Beschleunigungsensors und des Gyroskops fusionieren. So kann man 
alle
\\
\section {Bewegung}
Eine Interessante Bewegung stellt vor allem der Zugarm des 
Schüzten dar. Der Auszug verläuft nahezu linear, der häufigere Fehler 
an dieser Stelle versteckt sich allerdings in der Höhe des Zugarms. 
Um diese zu messen muss man die Erdanziehungskraft der Z-Achse herausrechnen.\\
Möglich ist ebenso die Neigung des Armes über die Euler-Winkel zu berechnen. Sollte 
der Arm zu hoch sein wird sich die Ausrichtung des Sensors stark verändern.\\
\\
Um den Schützen möglichst genau zu tracken sind mehrere Sensoren an eine Schützen denkbar und
wünschenswert. So könnten Fehler der einzelnen MPUs herausgerechnet werden.\\
Auch die klassifiezierung der Schüsse liefe so auf einem hörerem Standard.
\\




