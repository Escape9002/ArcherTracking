\section{Analyse}
Da der Schussablauf eines Bogenschützen viele Stationen mit verschiedenen Bewegungen 
beinhaltet, fällt es häufig sogar den Trainern schwer, zwischen einem technisch guten
oder schlechten Schuss zu unterscheiden.\\ 
Somit galt es einen Punkt zu finden, bei dem Fehler auffällig sind, so dass ich die Daten auch in 
der Praxis nachvollziehen kann.\\
\\
Der MPU9250 bietet 9 Freiheitsgrade (Degrees Of Freedom - DOF). Aus welchen diese bestehen, wird
im Kapitel zum MPU9250 erläutert. Diese Freiheitsgrade lassen es zu, die Orientierung des Sensors 
und damit die Orientierung des Körpers an der er befestigt ist, genau zu messen. Sogar die Distanz 
die der Sensor sich bewegt ist in der Theorie messbar.\\
\\
So ist eine Darstellung als 3D-Modell möglich, an der der Schütze seine Fehler sieht und Unterschiede
zu vorangegangen Schüssen hervorgehoben werden. \\
Denkbar wäre eine Klassifizierung der einzelnen Schüsse, um, wie in einem Videospiel, die Genauigkeit
der Wiederholung in Prozent anzugeben. Interessant wird es, sobald mehrere Schützen Datensätze
ihrer Schüsse vergleichen. Unterschiede werden ersichtlich, genau wie Gemeinsamkeiten.\\

\section{3D-Darstellung (Theorie)}
Um die Orientierung des Sensors zu berechnen geht man auf die Grundaussagen der Sensoren zurück. Der 
Beschleunigungssensor teilt mir die G-Kräfte mit, daraus lassen sich Rückschlüsse über die Orientierung 
der Pitch- und Roll-Achse ziehen. Das Gyroskop gibt Drehmoment-Werte zurück, worüber sich der Beschleunigungssensor
überprüfen lässt und eine Drehung um die Yaw-Achse messbar macht. Das Magnetomenter setzt schließlich alle Werte in
Relation zum Erdmagnetfeld und kann sowohl das Gyroskop als auch den Beschleunigungssensor überprüfen. Diese Überprüfung
findet über Filter statt, in meinem Projekt verwende ich hierfür den AHRS-Filter\footnote{Der AHRS-Filter 
ist ein erweiterter Kalman-Filter. Er passt seine Fehlerwerte über Zeit dem Drift der Sensorik an und bietet so auch 
auf lange Zeit gute Ergebnisse.}. \\
Für die Darstellung der Orientierung im Dreidimensionalen Raum gibt es mehrere Ansätze, die zwei bekanntesten 
sind wohl die Euler-Winkel und die Quaternionen. Die Vor- und Nachteile werden im folgenden erläutert.

\subsection{Euler-Winkel}
Bei Euler-Winkeln wird das Objekt um die einzelnen Achsen in einer festeglegten Reihenfolge gedreht. Sollten hierbei 
nach einer Drehung zwei Achsen aufeinander liegen, entseht der sogenannte \textit{Gimbal-Lock} und die Darstellung
der Bewegung findet fehlerhaft statt. Für reele Systeme wie Gimbal an einer Drohne ist dieser Fehler vernachlässigbar,
da er nur selten vorkommt und die Euler-Winkel den tatsächlich benötigten Drehwinkeln der Motoren entsprechen.
Da ich jedoch keine Motoren ansteuern muss, steht vor allem der Fehler im Mittelpunkt und ist der Grund, das ich nicht
die Euler-Winkel verwende.  

\subsection{Quaternionen}
Um die Position eines Körpers im dreidimensionalen Raum darzustellen reichen drei Koordinaten, 
die Möglichkeit der Orientierung verlangt jedoch nach einer vierten Koordinate und damit vier Dimensionen. 
Die Quaternionen stellen diese vierte Koordinate dar. Quaternionen finden damit im vierdimensionalen Raum statt
und sind damit für wenige Menschen vollständig nachvollziehbar. Ein Überprüfen der Rohdaten fällt hiermit für mich aus.
Allerdings schaffen es die Quaternionen durch diese weitere Dimension jegliche Orientierung im Raum darzustellen ohne
etwas ähnliches wie einen \textit{Gimbal-Lock} zu erzeugen. Die Vollständige Erklärung der Quaternionen sprengt momentan
leider meinen Wissenstand. Die einfache Eingabe meiner Daten in fast alle läufige 3D-Visualisierungsprogramme und 
der fehlende \textit{Gimbal-Lock} sind der Grund für mich, zukünftig die Quaternionen zu verwenden.

\section {Bewegung}
Eine interessante Bewegung stellt vor allem der Zugarm des 
Schützen dar. Der Auszug verläuft nahezu linear, der häufigere Fehler 
an dieser Stelle versteckt sich in der Höhe des Zugarms. 
Um diese zu messen, muss man die Erdanziehungskraft der Z-Achse herausrechnen.\\
Möglich ist ebenso, die Neigung des Armes über die Euler-Winkel zu berechnen. Sollte 
der Arm zu hoch sein, wird sich die Ausrichtung des Sensors stark verändern.\\
\\
Um den Schützen möglichst genau zu tracken, sind mehrere Sensoren an einem Schützen denkbar und
wünschenswert. So könnten Fehler der einzelnen MPUs herausgerechnet werden.
Auch die Klassifizierung der Schüsse liefe so auf einem höheren Standard.