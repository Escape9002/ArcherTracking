\chapter{Bewegungen, Analyse und bekannte Systeme}
\section{Analyse}
Da der Schussablauf eines Bogenschützen viele Stationen mit verschiedenen Bewegungen 
beinhaltet fällt es häufig sogar den Trainern schwer zwischen einem technisch guten
oder schlechtem Schuss zu unterscheiden.\\ 
Somit galt es einen Punkt zu finden bei dem Fehler auffällig sind, so das ich die Daten auch im 
Leben nachvollziehen kann.\\
\\
Der MPU9250 bietet 9 Freiheitsgrade (Degress Of Freedom - DOF). Aus welchen diese bestehen wird
im Kapitel zum MPU9250 erläutert. Diese Freiheitsgrade lassen es zu, die Orientierung des Sensors 
und damit die Orientierung des Körpers an der er befestigt ist genau zu messen. Sogar die Distanz 
die der Sensor sich bewegt ist messbar.\\
\\
So ist eine Darstellung als 3D-Modell möglich an der der Schütze seine Fehler sieht und Unterschiede
zu vorangegangen SChüssen hervorgehoben werden. \\
Denkbar wäre eine klassifiezierung der einzelnen Schüsse um wie in einem Videospiel die Genauigkeit
der Wiederholung in Prozent anzugeben. Interessant wird es, sobald mehrere Schützen Datensätze
ihrer Schüsse vergleichen. Unterschiede werden ersichtlich, genau wie Gemeinsamkeiten.
\\
\section {Bewegung}
Eine Interessante Bewegung stellt vor allem der Zugarm des 
Schüzten dar. Der Auszug verläuft nahezu linear, der häufigere Fehler 
an dieser Stelle versteckt sich allerdings in der Höhe des Zugarms. 
Um diese zu messen muss man die Erdanziehungskraft der Z-Achse herausrechnen.\\
\\
Um den Schützen möglichst genau zu tracken sind mehrere Sensoren an eine Schützen denkbar und
wünschenswert. So könnten Fehler der einzelnen MPUs herausgerechnet werden.\\
Auch die klassifiezierung der Schüsse liefe so auf einem hörerem Standard.
\\

\section{IMU vs Kameratracking}
Mein Projekt überschneidet sich in seinen Zielen häufig 
mit Tracking das bei VR-Brillen eingesetzt wird. Hier wird
zur Feststellung der Position des Spielers häufig eine 
Kombination aus Kameratracking und Infrarot-LED. \\
Hierbei muss der Spieler die Fernbedienungen festhalten die
die Infrarot-LEDs beinhalten, während die Kameras im Raum 
so verteilt werden müssen das der Spieler immer erkannt wird.\\
Die Neigung des Kopfes und der Hände werden auch hier häufig 
Mithilfe eines IMU bestimmt.\\
\\
Da diese Systeme viel Platz benötigen, viel Geld kosten und 
für die Bildverarbeitung häufig eine große Rechenkraft benötigen
ist dieses System nicht für viele Privatnutzer sinnvoll oder 
bieten einen Bewegungsfreiraum der Sport zu lässt.\\
\\
Die IMU-Sensoren bestehen mindestens aus einem Gyroskop und einem 
Beschleunigungssensor, manche bieten sogar ein Magnetometer an.
Somit sollte es möglich sein, über die
Beschleunigung die Distantz die ein Körper mit diesem Sensor zurück
legt zu messen. Die Neigung und Orientation sind über das Gyroskop 
und Magnetometer sehr genau messbar.\\
\\
Die Vorteile der IMU liegen auf der Hand, sie sind günstig, klein
und leicht. Aus diesen Gründen trägt fast jeder heutzutage 
so einen Sensor bei sich, die meisten Handys haben ihn schon eingebaut.\\
\\
Für mein Projekt benutze ich dennoch einen eigenen IMU, um die
Qualität der Messdaten sicher zu stellen.

