\section{Analyse}
Da der Schussablauf eines Bogenschützen viele Stationen mit verschiedenen Bewegungen 
beinhaltet, fällt es häufig sogar den Trainern schwer, zwischen einem technisch guten
oder schlechten Schuss zu unterscheiden.\\ 
Somit galt es einen Punkt zu finden, bei dem Fehler auffällig sind, so dass ich die Daten auch in 
der Praxis nachvollziehen kann.\\
\\
Eine interessante Bewegung stellt vor allem der Zugarm des 
Schützen dar. Der Auszug verläuft nahezu linear, der häufigere Fehler 
an dieser Stelle versteckt sich in der Höhe des Zugarms. 
Um diese zu messen, muss man die Erdanziehungskraft der Z-Achse herausrechnen.\\
Möglich ist ebenso, die Neigung des Armes über die Euler-Winkel zu berechnen. Sollte 
der Arm zu hoch sein, wird sich die Ausrichtung des Sensors stark verändern, so wie 
die zurückgelegte Distanz in Z-Richtung.\\
\\
Um den vollen Bewegungsablauf eines Schützen zu tracken, wird an jedem Körperteil ein Sensor
benötigt. Dies ermöglicht nicht nur das gegenseitige Überprüfen auf Sinnhaftigkeit der Orientierung
(menschliches Skelett hat beschränkte Bewegungsfreiräume), sondern auch die Klassifizierung der 
einzelnen Schuss-Abläufe könnte, auf Grund von mehr Daten, genauer ablaufen.\\ 
\\
Der MPU9250 bietet 9 Freiheitsgrade (Degrees Of Freedom - DOF). Aus welchen diese bestehen, wird
im Kapitel zum MPU9250 erläutert. Diese Freiheitsgrade lassen es zu, die Distanz
die der Sensor sich in alle Richtungen bewegt, zumindest in der Theorie, zu berechnen. Die Orientierung des Sensors 
und damit die Orientierung des Körpers an der er befestigt ist, kann man genau messen und berechnen. \\
\\
So ist eine Darstellung als 3D-Modell möglich, an der der Schütze seine Fehler sieht und Unterschiede
zu vorangegangen Schüssen hervorgehoben werden. \\
Denkbar wäre eine Klassifizierung der einzelnen Schüsse, um, wie in einem Videospiel, die Genauigkeit
der Wiederholung in Prozent anzugeben. Interessant wird es, sobald mehrere Schützen Datensätze
ihrer Schüsse vergleichen. Unterschiede werden ersichtlich, genau wie Gemeinsamkeiten. \\
\\

Dieses Kapitel beschäftigt sich im Folgenden mit der dreidimensionalen Visualisierung.

\section{Integrale}
Die Distanz, berechenbar aus den Werten des Beschleunigungssensors, kann uns die Differenz zwischen einem Startpunkt
und dem jetzigen Punkt des Sensors berechnen. Durch die drei Achsen meines Beschleunigungssensors ginge dies sogar in alle Richtungen. Damit 
könnte man ein Modell erzeugen, das den betreffenden Punkt in die einzelnen Richtungen verschiebt, bis der finale
neue Standort des Sensors erreicht wird. \\ 
\\ 
Um die Distanz aus der gemessenen Beschleunigung zu berechnen,
fand ich zwei verschiedene Formeln. Die wohl bekannteste Umrechnung 
benutzt Integrale, die zweite Formel ist die der gleichm"aßigen 
Beschleunigung.\\
\\
Die Integration wird von allen mir bekannten Forschungen verwendet. 
Man muss eine Doppelintegration
ausführen um von Beschleunigung auf Distanz zu kommen, hierbei verwandelt 
sich das Rauschen des Sensors in Drift, so entsteht einen exponentiell steigender
Fehler. \\ 
Die gleichmäßige Formel kann, im Gegensatz zum Integral, nur positive 
Beschleunigungen verwerten, hat in den Tests allerdings 
deutlich genauere Werte und einen geringeren Fehler bei Stillstand 
aufgezeigt. \\
\\
Als Zeit wird die Frequenz, mit der der Sensor Daten misst, 
genommen. Hierfür wird die Frequenz in Zeitabschnitte umgerechnet, mit 
der die Formeln letzendlich arbeiten. Der Testaufbau und die Durchführung 
ist im \textit{ 7.1 Distanz-Testaufbau} zu sehen. Die Formel entspricht hierbei der 
Umrechnung in mHz, da ein Skalierungsfehler bei der Verwendung der normalen \textit{1/f}-Formel entsteht.
\subsection{Gleichmäßige Formel}
Die angepasste Formel f"ur gleichm"aßige Beschleunigung berechnete
die Distanz in meinen Versuchen mit einer Genauigkeit von
$\pm8$ cm auf 30cm Teststrecke.\\
Die physikalische Formel für Bewegungen mit einer Anfangsgeschwindigkeit lautet 
$s + v * t  + a * t^2 * 0.5$,
allerdings wurden mit dem leicht verändertem Weg-Zeit-Gesetz $s + a * t^2 * 0.5$ 
bessere Testergebnisse erzielt.\\
Da die negative Beschleunigung den Wert der Distanz wieder nullierte,
wurden zu dieser Formel nur positive Distanzmessungen zugelassen.\\
Hier der Code-Ausschnitt:
\begin{verbatim}
    if (acc > 0) {
    t = (freq / 1000); //hz is not time but frequenzy

    distance = (distance) + (acc * (t * t) * 0.5);
    velocity = acc * t;
    }
\end{verbatim}
F"ur die Tests wurde dieser Code statt der BLE-"Ubetragung auf dem Arduino ausgef"uhrt. 

\subsection{Integral}
Die Berechnung der Distanz über Integrale ist der Standard 
in der Wissenschaft. Beim integrieren meiner Sensordaten ist zu beachten, 
das beim mathematischen Integrieren die zu berechnenden Werte gegen den Limes 
laufen können. Da ich nur begrenzete Datenmengen zur Verfügung habe, ist dies 
nicht möglich, wodurch allein das Integrieren Fehler erzeugt. Ebenfalls wird das 
Rauschen meines Sensors nicht beachtet und aufaddiert. Da ich zwei mal integrieren 
muss, ist dieser Fehler in der Distanz exponentiell.\\
\\
Die Testergebnisse ergaben einen Fehler von $\pm15$cm auf einer Strecke
von 30cm. Ebenso war das Ergebnis sofort in Zentimetern, statt wie erwartet in Metern. 
Wurde der Sensor zurückbewegt an seinen Startpunkt, sank
der Wert jedoch wieder auf Null ab. Somit kann man schließen,
dass das Ergebniss nur falsch skaliert ist.\\
Dieser Fehler wurde noch nicht behoben. \\
Ein klarer Vorteil dieser Rechnung zeigt sich schon beim Test,
die Formel funktioniert auch für negative Beschleunigungen.
Der Wert sinkt am Ende der Teststrecke nicht ab, sondern bleibt
auf seinem hohen Wert.\\
Folgend der Code-Ausschnitt der Integral-Rechnung:
\begin{verbatim}
    t = (freq / 1000); // hz to time

    velocity = t * ((acc + accOld) / 2) + velocity;
    accOld = acc;
    distance = t * ((velocity + velocityOld) / 2) + distance;
    velocityOld = velocity;  
\end{verbatim}

\section{Orientierungen}
Um die Orientierung des Sensors zu berechnen, geht man auf die Grundaussagen der Sensoren zurück. Der 
Beschleunigungssensor gibt die G-Kräfte an, daraus lassen sich Rückschlüsse über die Orientierung 
der Pitch- und Roll-Achse ziehen. Das Gyroskop gibt Drehmoment-Werte zurück, worüber sich der Beschleunigungssensor
überprüfen lässt und eine Drehung um die Yaw-Achse messbar macht. Das Magnetometer setzt schließlich alle Werte in
Relation zum Erdmagnetfeld und kann sowohl das Gyroskop als auch den Beschleunigungssensor überprüfen. Diese Überprüfung
findet über Filter statt, in meinem Projekt verwende ich hierfür den AHRS-Filter\footnote{Der AHRS-Filter 
ist ein erweiterter Kalman-Filter. Er passt seine Fehlerwerte über Zeit dem Drift der Sensorik an und bietet so auch 
auf lange Zeit gute Ergebnisse.}. Der AHRS-Filter fusioniert die Daten in eine verwendbar Orientierung.\\
Für die Darstellung der Orientierung im dreidimensionalen Raum gibt es mehrere Ansätze, die zwei bekanntesten 
sind wohl die Euler-Winkel und die Quaternionen. Die Vor- und Nachteile werden im Folgenden erläutert.
\subsection{Euler-Winkel}
Bei Euler-Winkeln wird das Objekt um die einzelnen Achsen in einer festeglegten Reihenfolge gedreht. Sollten hierbei 
nach einer Drehung zwei Achsen aufeinander liegen, entseht der sogenannte \textit{Gimbal-Lock} und die Darstellung
der Bewegung findet fehlerhaft statt. Für reele Systeme wie Gimbal an einer Drohne ist dieser Fehler vernachlässigbar,
da er nur selten vorkommt und die Euler-Winkel den tatsächlich benötigten Drehwinkeln der Motoren entsprechen.
Da ich jedoch keine Motoren ansteuern muss, steht vor allem der Fehler im Mittelpunkt und ist der Grund, das ich nicht
die Euler-Winkel verwende.  

\subsection{Quaternionen}
Um die Position eines Körpers im dreidimensionalen Raum darzustellen reichen drei Koordinaten, 
die Möglichkeit der Orientierung verlangt jedoch nach einer vierten Koordinate und damit vier Dimensionen. 
Die Quaternionen stellen diese vierte Koordinate dar. Quaternionen finden damit im vierdimensionalen Raum statt
und sind damit für wenige Menschen vollständig nachvollziehbar. Ein Überprüfen der Rohdaten fällt hiermit für mich aus.
Allerdings schaffen es die Quaternionen durch diese weitere Dimension jegliche Orientierung im Raum darzustellen ohne
etwas ähnliches wie einen \textit{Gimbal-Lock} zu erzeugen. Die Vollständige Erklärung der Quaternionen sprengt momentan
leider meinen Wissenstand. Die einfache Eingabe meiner Daten in fast alle läufige 3D-Visualisierungsprogramme und 
der fehlende \textit{Gimbal-Lock} sind der Grund für mich, zukünftig die Quaternionen zu verwenden.