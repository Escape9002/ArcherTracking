Die Darstellung erfolgt momentan über Matlab und OpenSim. In Matlab werden die IMU-Daten
fusioniert, die Orientierung in Quaternionen berechnet und in das korrekte Daten-Format geschrieben. 
OpenSim schreibt den Sensor-Daten dann den angegeben Knochen eines Skellets zu und Orientiert diese anhand 
der Orientierungsdaten neu.

\subsection{Matlab und OpenSim}
Matlab ist ein kostenpflichtiges Programm und wurde mir von meiner Schule im Rahmen der MakerSpace-AG zur Verfügung 
gestellt. Ich benutze es, um einerseits Live-Daten des Sensors zu verwerten als auch um die gespeicherten Handy-Daten 
umzuformen und für spätere Benutzung vorzubereiten. Matlab stellt mir hierbei eine einfache verwendung des AHRS-Filters
bereit. Der AHRS-Filter berechnet hierbei die Orientierung in Quaternionen. \\ 
Die Sensorwerte werden für OpenSim in das Datenformat \textit{.sto} umgeschrieben. OpenSim ließt aus dieser Datei die 
manuell zugewiesenen Messpunkte wie Ober-/Unterarm aus und weißt die folgenden Orientierungen mit zeitlicher Kennnummer
den verschiedenen Körperteilen zu. Mithilfe von Inversivers Kinematik berechnet OpenSim den aufgenommenen Bewegungsablauf.
Neben den Sensoren zur Bewegungsaufnahme empfiehlt es sich, einen Sensor an der Hüfte zu platzieren. Dieser dient durch
wenig Bewegung in jegliche Richtung als Null-Punkt und kann als Relationspunkt für die anderen Sensoren verwendet werden.

\subsection{Inversive Kinematik}
In Arbeit...