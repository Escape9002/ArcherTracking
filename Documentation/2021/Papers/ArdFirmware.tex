Das Programm auf dem Arduino Nano 33 Ble stellt zu Beginn eine 
I$^2$C Verbindung mit dem MPU9250 her. Diese Verbindung wird über
die Wire-Bibiliothek im Fast-Mode (Frequenz:400 000) hergestellt. 
Das BLE-Objekt zur kommunikation
mit dem Boardeigenem Chip wird ebenfalls initialisiert. \\
Sobald eine BLE-VErbindung steht, fragt der Arduino
den Sensor ab. Sollte dieser neue Daten 
bereitgestellt haben, werden die Daten in einem String 
verbunden und über die zu den Daten gehörende Characteristic 
veröffentlicht. Dabei ist die Characteristic so eingestellt das 
verbundene Geräte eine Nachricht bekommen, sobald der Wert 
der Characteristic sich verändert.\\
\\
Der Motion-Processor wird von der Bibliothek verwendet und 
kann so die Roh-Daten des Sensors mit einem Tief-Pass-Filter
vorverarbeiten. Weitere Datenverarbeitung übernimmt der 
Arduino nicht, um eine möglichst hohe Datenrate zu garantieren.