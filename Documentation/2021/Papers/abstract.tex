Kurzfassung:\\
\small
Mithilfe eines Beschleunigungssensors kann man viele Bewegungen erforschen und vermessen. 
Mit meinem System sollen ''IMU'' dazu eingesetzt werden, Bewegungsabläufe aufzunehmen, zu erkennen und miteinander zu vergleichen.\\
Die Daten werden über Bluetooth-Low-Energy an ein Handy geschickt, wo sie angesehen und gespeichert 
werden können. Im Unterschied zum Vorjahr ist es mir nun auch möglich, die Bewegung dreidimensional 
darzustellen und somit die Auswertung benutzerfreundlich zu gestalten. Die Visualisierung findet an 
einem PC statt.\\
Mein System soll die Verwendung von Kameras bei Bewegungserfassungen ersetzen oder erweitern, da diese Systeme 
nicht nur viel Platz benötigen, sondern sehr teuer sein können und tote Winkel besitzen. 
Durch einen niedrigen Preis, kleine Formate und eine einfache Handhabung soll die Bewegungsanalyse 
so in den Alltag rücken. Für mich steht als Anwendungsfall ein wiederholungsträchtiger Sport wie 
Bogenschießen im Mittelpunkt. Mein System soll es jedem\footnote{Aus Gründen der Lesbarkeit wird das Gendern weggelassen} 
ermöglichen den Bewegungsablauf ohne Trainer zu trainieren.\\
Auch das Erkennen von Krankheiten im Bewegungsapparat stellt einen Anwendungsbereich dar, auf den ich mich jedoch nicht konzentrierte.