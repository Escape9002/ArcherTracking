Die Genauigkeit der gemessenen Distanz des MPU9250 hat zu hohe Fehlerwerte, als 
dass sie momentan ohne Vorverarbeitung im dreidimensionalen Tracking verwendet werden könnten.
Allerdings sind momentan keine Filter oder Fusionen der Sensoren des MPU9250 implementiert,
weshalb eine Steigerung der Messgenauigkeit noch möglich ist. Es gibt Forschungen auf diesem Gebiet,
die akzeptable Ergebnisse erreichten. Im einher zu diesen Erfolgen kamen jedoch immer Einschränkungen 
die Benutzung betreffend, die mich schließlich von der Idee abbrachten, in eine ähnliche Richtung, 
im Rahmen dieses Projekts, zu forschen.\\
Die Orientierung hingegen kann genau berechnet werden, was die Visualisierung ermöglicht.\\
Die Berechnung der Distanz, wie anfangs geplant, ist nicht ohne erheblichen Aufwand möglich.
Somit fällt die Idee darüber die Gliedmaßen zu tracken aus. Stattdessen wird durch Drittsoftware
die Orientierung berechnet und verwendet.\\
Hierbei trägt die Wahl der Sportart der Funktionnstüchtigkeit des Projekts bei. Beim Bogenschießen
findet wenig Bewegung statt, die nicht über eine Neu-Orientierung dargestellt werden kann, weshalb die
Berechnung der Distanz für dieses Projekt nicht nötig ist.