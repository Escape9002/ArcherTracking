Um die Distanz aus der gemessenen Beschleunigung zu berechnen
gibt es 2 mögliche Wege. \\
\\
Der erste Weg berechnet die Distanz 
mittels der Formel für gleichmäsig Beschleunigte Bewegungen.
Die gemessene Bewegung ist allerdings nicht gleichmäßig, womit
dieser Ansatz prinzipeill falsch ist.\\
Allerdings lieferte diese Formel die besseren Ergebnisse 
bei Tests.\\
\\
Der zweite Weg verwendet Integrale, ein Ansatz der auch von
verschiedenen anderen Forschern auf diesem Gebiet verfolgt wird.
Um von Beschleunigung auf Distanz zu kommen muss man zwei mal 
integrieren.\\
Das Problem das schon beim ersten Integral auftritt ist, 
das Mess-Fehler von sogennanten Rauschen zu Drift wechseln.
Dieser Fehler verstärkt sich beim zweiten integrieren weiter 
und liefert so schnell ungenaue Werte.\\