Um die Distanz aus der gemessenen Beschleunigung zu berechnen,
fand ich zwei verschiedene Formeln. Die wohl bekannteste Umrechnung 
benutzt Integrale, die zweite Formel ist die der gleichm"aßigen 
Beschleunigung.\\
\\
Die Integration wird von allen mir bekannten Forschungen verwendet. 
Man muss eine Doppelintegration
ausführen um von Beschleunigung auf Distanz zu kommen, hierbei verwandelt 
sich das Rauschen des Sensors in Drift und so einen exponentiell steigenden
Fehler. \\ 
Die gleichmäßige Formel kann im Gegensatz zum Integrall, nur positive 
Beschleunigungen verwerten, hat in den folgenden Tests allerdings 
deutlich genauere Werte und einen geringeren Fehler bei Stillstand 
aufgezeigt. So wird in diesem Projekt die gleichmäßige Beschleunigungsformel 
verwendet.\\
\\
Als Zeit wird die Frequenz, mit der der Sensor Daten misst, 
genommen. Hierfür wird die Frequenz in Zeitabschnitte umgerechnet, mit 
der die Formeln letzendlich arbeiten.