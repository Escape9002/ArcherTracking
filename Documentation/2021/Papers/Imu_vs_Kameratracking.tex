Mein Projekt überschneidet sich in seinen Zielen häufig 
mit Tracking das bei VR-Brillen eingesetzt wird. Hier wird
zur Feststellung der Position des Spielers häufig eine 
Kombination aus Kameratracking und Infrarot-LED. \\
Hierbei muss der Spieler die Fernbedienungen festhalten die
die Infrarot-LEDs beinhalten, während die Kameras im Raum 
so verteilt werden müssen das der Spieler immer erkannt wird.\\
Die Neigung des Kopfes und der Hände werden auch hier häufig 
Mithilfe eines IMU bestimmt.\\
\\
Da diese Systeme viel Platz benötigen, viel Geld kosten und 
für die Bildverarbeitung häufig eine große Rechenkraft benötigen
ist dieses System nicht für viele Privatnutzer sinnvoll oder 
bieten einen Bewegungsfreiraum der Sport zu lässt.\\


%\\
%Die IMU-Sensoren bestehen mindestens aus einem Gyroskop und einem 
%Beschleunigungssensor, manche bieten sogar ein Magnetometer an.
%Somit sollte es möglich sein, über die
%Beschleunigung die Distantz die ein Körper mit diesem Sensor zurück
%legt zu messen. Die Neigung und Orientation sind über das Gyroskop 
%und Magnetometer sehr genau messbar.\\
%\\
%Die Vorteile der IMU liegen auf der Hand, sie sind günstig, klein
%und leicht. Aus diesen Gründen trägt fast jeder heutzutage 
%so einen Sensor bei sich, die meisten Handys haben ihn schon eingebaut.\\
%\\
%Für mein Projekt benutze ich dennoch einen eigenen MPU, um die
%Qualität der Messdaten sicher zu stellen.
