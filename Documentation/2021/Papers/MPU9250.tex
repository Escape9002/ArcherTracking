Für eine genaue Datenlage sorgt in meinem Projekt der Multi-Chip MPU9250. Dieser ist mit 
einem 3-Achsen Beschleunigungssensor, einem 3- Achsen Gyroskop Sensor und einem 3-
Achsen Magnetometer ausgerüstet. Alle Sensoren beinhalten eine Kalibrierung innerhalb der 
Firma und einen Selbsttest bei Benutzung. Der Sensor benötigt nur 2,4 bis 3,6 Volt für die 
Inbetriebnahme. In der Tabelle sind die Datenpins und die Genauigkeit der Sensoren notiert.\\
\\
\begin{tabularx}{0.8\textwidth}{l|X}
Eigenschaft & Daten \\
\hline
Memory & 1MB Flash | 256 KB SRAM \\ 

Interfaces & $I^2C$, SPI, \dots \\

Volt & Input: 4,5 - 21 V | Output: 3,3 V\\
\end{tabularx}
\\
\\
Die Daten werden über den I²C-Bus vom Arduino abgefragt. Die Abtastrate beträgt hierbei 
stolze mögliche 400kHz. Dabei werden alle Sensoren abgefragt und die Daten versendet.
Die geringe Größe des Chips (150*250), der geringe Energieverbrauch (3,5mA wenn 
alle Sensoren ausgelesen werden), die hohe Genauigkeit und Geschwindigkeit der 
Datenübertragung lassen diesen Sensor perfekt für dieses Projekt werden. Der verbaute
DMP (Digital Motion Processor) wird ebenfalls verwendet und filtert die Daten mit einem
Low-Pass-Filter.