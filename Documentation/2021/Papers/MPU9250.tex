Für eine genaue Datenlage sorgt in meinem Projekt der Multi-Chip MPU9250. Dieser ist mit 
einem 3-Achsen Beschleunigungssensor, einem 3-Achsen Gyroskop Sensor und einem 3-Achsen 
Magnetometer ausgerüstet. Damit bietet er neun Freihtsgrade (9 Degrees of Freedom).
Alle Sensoren erhielzen eine Kalibrierung innerhalb der 
Firma und können einen Selbsttest bei Benutzung vollziehen. Der Sensor benötigt nur 2,4 bis 3,6 Volt während 
des Betriebs. In der Tabelle sind die Datenpins und die Genauigkeit der Sensoren notiert.\\
\\
\begin{tabularx}{0.8\textwidth}{l|X|XX}
Sensoren & Datenübertragung & Empfindlichkeit                                     \\
\hline
Gyroskop & 3 * 16bit ADCs & $\pm250°/\sec$, $\pm500\sec$, $\pm1000°/\sec$, $\pm2000°/\sec$\\ 
\hline
Beschleunigungssensor & 3 * 16bit ADCs & $\pm2g$, $\pm4g$, $\pm8g$, $\pm16g$\\
\hline
Magnetometer & 3 * 16bit ADCs & full-scale range of $\pm$\SI{4800}{\milli\tesla\meter}T \\
\hline
Übertragung & $I^2C$, SPI, \dots & \\
\end{tabularx}
\\
\\
Die Daten werden über den I²C-Bus vom Arduino abgefragt. Die Abtastrate beträgt hierbei 
mögliche 400kHz. Dabei werden alle Sensoren abgefragt und die Daten versendet.
Durch die geringe Größe des Chips (150mm*250mm), dem geringem Energieverbrauch (3,5mA wenn 
alle Sensoren ausgelesen werden), der hohen Genauigkeit und der Geschwindigkeit der 
Datenübertragung ist dieser Sensor perfekt für dieses Projekt. Der verbaute
DMP (Digital Motion Processor) wird ebenfalls verwendet und filtert die Daten mit einem
Low-Pass-Filter.

