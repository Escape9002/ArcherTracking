Der Stromverbrauch wurde mit einem Multimeter am Batterieanschluss in
verschiedenen Modi gemessen. Die Ergebnisse stehen in der Tabelle:\\
%Tabelle: Werte: 
%200mA Auslösung:
%BLE aus: 5
%BLE an: 10.2
%BLE Verbunden: 13
%MPU9250 an: 5
%Bereich: 200 | Auflösung : 0,1 uA | +-1% des Messwerts +- 2 Ziffern
%Quelle:https://github.com/Escape9002/ArcherTracking/issues/5
\\
\begin{tabularx}{0.8\textwidth}{l|X|XX}
    Modul & An/Aus & Verbrauch(in mA) \\
    \hline
    BLE & aus & 5 \\
    \hline
    BLE & an & 10.2 \\
    \hline
    BLE & an und verbunden & 13 \\
    \hline
    MPU9250 & an & 5 \\
\end{tabularx}\\
\\
Der Stromverbrauch lässt sich so berechnen und erleichtert die korrekte Batterie-Wahl.
Die verwendeten Formeln: 
\begin{equation}
    $$
    P = U / I \\
    Ah / V = Wh \\
    Wh / W = t $\rightarrow$ (U*I*t) / U*I = t \\
    $$
\end{equation}
So hat der Arduino eine Leistungaufnahme von:
\begin{equation}
    $$
    0,005 A * 7,4 V = 0,037 W
    $$
\end{equation}
\\
Die verschiedenen Batterien-Typen stehen in der folgenden Tabelle:\\
\\
\begin{tabularx}{0.8\textwidth}{l|X|X|X|XX}
    Typ & Laufzeit(in Stunden) & Gewicht & Cut-Off-Spannung & Differenz\\
    \hline
    9V & 40 & 50g &7.2V & $9-7.2 = 2,8$\\
    \hline
    CR2025 & 12 & 2,5g & 2V & $3-2 = 1$\\
    \hline
    2 * CR2025 & 48,6 & 5g & 2V & $6-2 = 4$\\
\end{tabularx}\\
%Tabelle
%9V : 40h, 0,05 KG, CutoffVol: 7.2V (9-7.2 = 2,8)
% CR2025 : 12h , 0,0025 Kg, CutoffVolt: 2V (3-2 = 1)
% CR2025 * 2 = 48,6 h, 0,005 Kg, CutoffVolt: 2V (6-2 = 4)
\\
Um eine Stromversorgung des Arduinos sicher zu stellen, benötigt man 2 Knopfzellen
des Typs CR2025. Dennoch hat die Knopfzelle CR2025 nicht nur eine bessere 
Laufzeit sondern ebenfalls weniger Gewicht, braucht weniger Platz und eine bessere 
Differenz zwischen Cut-Off-Spannung zu angebotener Spannung.
\\
Gegen das Umrüsten auf die Knopfbatterie spricht einzig der Umweltschutz.
Denn im Gegensatz zu 9-Volt-Batterien gibt es keine Akkus für Knopfzellen.