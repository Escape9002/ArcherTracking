Der Stromverbrauch wurde mit einem Multimeter am Batterieanschluss in
verschiedenen Modi gemessen. Die Ergebnisse stehen in der Tabelle:\\
%Tabelle: Werte: 
%200mA Auslösung:

%BLE aus: 5
%BLE an: 10.2
%BLE Verbunden: 13
%MPU9250 an: 5
%Bereich: 200 | Auflösung : 0,1 uA | +-1% des Messwerts +- 2 Ziffern
%Quelle:https://github.com/Escape9002/ArcherTracking/issues/5

Der Stromverbrauch lässt so berechnen und erleichter die Korrekte Batterie-Wahl.
\\
Die Verwendeten Formeln: 
P = U / I\\
Ah / V = Wh\\
Wh / W = t --> (U*I*t) / U*I = t\\
\\
So verbraucht der Arduino \\
0,005 A * 7,4 V = 0,037 W\\
\\
Die Verschiedenen Batterien-Typen stehen in der folgenden Tabelle:\\
%Tabelle
%9V : 40h, 0,05 KG, CutoffVol: 7.2V (9-7.2 = 2,8)
% CR2025 : 12h , 0,0025 Kg, CutoffVolt: 2V (3-2 = 1)
% CR2025 * 2 = 48,6 h, 0,005 Kg, CutoffVolt: 2V (6-2 = 4)
\\
Um eine Stromversorgung des Arduinos sicher zu stellen benötigt man 2 Knopfzellen
des Typs CR2025.Dennoch hat die Knopfzelle CR2025 nicht nur eine bessere 
Laufzeit sondern ebenfalls weniger Gewicht, weniger Platz und eine bessere 
Differenz zwischen Cut-Off-Spannung zu angenotener Spannung.
\\
Gegen das umrüsten auf die Knopfbatterie spricht einzig der Umweltschutz. 
Denn im Gegensatz zu 9-Volt-Batterien gibt es keine Akkus für Knopfzellen.