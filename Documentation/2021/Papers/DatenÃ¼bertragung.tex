Die Daten werden, wie im Kapitel zur BLE-Datenübetragung beschrieben, in maximal
drei verschiedenen \textit{Charakteristiken} gesendet und empfangen. \\
Eine Instanz muss hierbei sicherstellen, dass Daten
nicht doppelt gesendet oder empfangen werden.\\
Dies wird beim Arduino durch die Abfrage einer bibliothekseigenen Funktion
sichergestellt, die erst auslöst, wenn neue Daten des MPU9250 erzeugt wurden. Dies bewirkt
das Bereitstellen der Daten in den Characteristics.\\
Sobald diese Daten gesendet wurden, bekommt das Handy ein Signal und liest
daraufhin die neuen Daten.
