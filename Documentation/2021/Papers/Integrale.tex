\subsection{Gleichmäßige Formel}
Die angepasste Formel f"ur gleichm"aßige Beschleunigung berechnete
die Distanz in meinen Versuchen mit einer Genauigkeit von
$\pm8$ cm auf 30cm Teststrecke.\\
Die physikalische Formel für Bewegungen mit einer Anfangsgeschwindigkeit lautet 
$s + v * t  + a * t^2 * 0.5$,
allerdings wurden mit dem leicht verändertem Weg-Zeit-Gesetz $s + a * t^2 * 0.5$ 
bessere Testergebnisse erzielt.\\
Da die negative Beschleunigung den Wert der Distanz wieder nullierte,
wurden zu dieser Formel nur positive Distanzmessungen zugelassen.\\
Hier der Code-Ausschnitt:
\begin{verbatim}
    if (acc > 0) {
    t = (freq / 1000); //hz is not time but frequenzy

    distance = (distance) + (acc * (t * t) * 0.5);
    velocity = acc * t;
    }
\end{verbatim}
F"ur die Tests wurde dieser Code statt der BLE-"Ubetragung auf dem Arduino ausgef"uhrt. 

\subsection{Integral}
Die Berechnung der Distanz über Integrale ist der Standard 
in der Wissenschaft. Beim integrieren meiner Sensordaten ist zu beachten, 
das beim mathematischen Integrieren die zu berechnenden Werte gegen den Limes 
laufen können. Da ich nur begrenzete Datenmengen zur Verfügung habe, ist dies 
nicht möglich, wodurch allein das Integrieren Fehler erzeugt. Ebenfalls wird das 
Rauschen meines Sensors nicht beachtet und aufaddiert. Da ich zwei mal integrieren 
muss, ist dieser Fehler in der Distanz exponentiell.\\
\\
Die Testergebnisse ergaben einen Fehler von $\pm15$cm auf einer Strecke
von 30cm. Ebenso war das Ergebnis sofort in Zentimetern, statt wie erwartet in Metern. 
Wurde der Sensor zurückbewegt an seinen Startpunkt, sank
der Wert jedoch wieder auf Null ab. Somit kann man schließen,
dass das Ergebniss nur falsch skaliert ist.\\
Dieser Fehler wurde noch nicht behoben. \\
Ein klarer Vorteil dieser Rechnung zeigt sich schon beim Test,
die Formel funktioniert auch für negative Beschleunigungen.
Der Wert sinkt am Ende der Teststrecke nicht ab, sondern bleibt
auf seinem hohen Wert.\\
Folgend der Code-Ausschnitt der Integral-Rechnung:
\begin{verbatim}
    t = (freq / 1000); // hz to time

    velocity = t * ((acc + accOld) / 2) + velocity;
    accOld = acc;
    distance = t * ((velocity + velocityOld) / 2) + distance;
    velocityOld = velocity;  
\end{verbatim}