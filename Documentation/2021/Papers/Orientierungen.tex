\subsection{Euler-Winkel}
Bei Euler-Winkeln wird das Objekt um die einzelnen Achsen in einer festeglegten Reihenfolge gedreht. Sollten hierbei 
nach einer Drehung zwei Achsen aufeinander liegen, entseht der sogenannte \textit{Gimbal-Lock} und die Darstellung
der Bewegung findet fehlerhaft statt. Für reele Systeme wie Gimbal an einer Drohne ist dieser Fehler vernachlässigbar,
da er nur selten vorkommt und die Euler-Winkel den tatsächlich benötigten Drehwinkeln der Motoren entsprechen.
Da ich jedoch keine Motoren ansteuern muss, steht vor allem der Fehler im Mittelpunkt und ist der Grund, das ich nicht
die Euler-Winkel verwende.  

\subsection{Quaternionen}
Um die Position eines Körpers im dreidimensionalen Raum darzustellen reichen drei Koordinaten, 
die Möglichkeit der Orientierung verlangt jedoch nach einer vierten Koordinate und damit vier Dimensionen. 
Die Quaternionen stellen diese vierte Koordinate dar. Quaternionen finden damit im vierdimensionalen Raum statt
und sind damit für wenige Menschen vollständig nachvollziehbar. Ein Überprüfen der Rohdaten fällt hiermit für mich aus.
Allerdings schaffen es die Quaternionen durch diese weitere Dimension jegliche Orientierung im Raum darzustellen ohne
etwas ähnliches wie einen \textit{Gimbal-Lock} zu erzeugen. Die Vollständige Erklärung der Quaternionen sprengt momentan
leider meinen Wissenstand. Die einfache Eingabe meiner Daten in fast alle läufige 3D-Visualisierungsprogramme und 
der fehlende \textit{Gimbal-Lock} sind der Grund für mich, zukünftig die Quaternionen zu verwenden.