\subsection{Euler-Winkel}
Bei Euler-Winkeln wird das Objekt um die einzelnen Achsen in einer festgelegten Reihenfolge gedreht. Sollten hierbei 
nach einer Drehung zwei Achsen aufeinander liegen, entsteht der sogenannte \textit{Gimbal-Lock} und die Darstellung
der Bewegung findet fehlerhaft statt. Für reele Systeme, wie Gimbal an einer Drohne, ist dieser Fehler vernachlässigbar,
da er nur selten vorkommt und die Euler-Winkel den tatsächlich benötigten Drehwinkeln der Motoren entsprechen.
Da ich jedoch keine Motoren ansteuern muss, steht vor allem der Fehler im Mittelpunkt und ist der Grund dafür, das ich nicht
die Euler-Winkel verwende.  

\subsection{Quaternionen}
Um die Position eines Körpers im dreidimensionalen Raum darzustellen, reichen drei Koordinaten, 
die Möglichkeit der Orientierung verlangt jedoch nach mindestens einer vierten Koordinate. 
Quaternionen beschreiben ein solches vierdimensionales System.
Die Formel der Quaternionen sieht eine Reelle und drei imaginäre Zahlen vor.
Die imaginären Zahlen im Verhältnis zueinander stehen und ihrere Werte, vereinfach gesagt, skaliert werden können, kann 
ein ''Richtungsvektor'' berechnet werden. Dieser kann in alle Richtungen zeigen. Das Objekt wird an diesem Vektor orientiert.
Ein Überprüfen der Rohdaten fällt durch die vierte Dimension für mich aus.
Allerdings schaffen es die Quaternionen durch diese weitere Dimension jegliche Orientierung im Raum darzustellen ohne
etwas ähnliches wie einen \textit{Gimbal-Lock} zu erzeugen. Die Vollständige Erklärung der Quaternionen sprengt momentan
leider meinen Wissenstand. Die einfache Eingabe meiner Daten in fast alle bekannte 3D-Visualisierungsprogramme und 
der fehlende \textit{Gimbal-Lock} sind der Grund für mich, zukünftig die Quaternionen zu verwenden.