Der Arduino Nano 33 BLE ist wie der Name schon sagt ein Prozessor aus dem Hause Arduino
der Nano Reihe. Was diesen von der normalen Nano-Reihe unterscheidet ist der BLE-Chip
NINA-b3(nRF52840) auf seinem Rücken. Dieser Chip ermöglicht es dem Arduino über 
Bluetooth 5.0, auch genannt Bluetooth-Low-Energy, mit allen anderen Bluetooth-Geräten ab 
der Bluetooth Version 4.0 kabellos zu kommunizieren.\\
\\
\begin{tabularx}{0.8\textwidth}{l|X|XX}
Sensoren & Datenübertragung & Empfindlichkeit                                     \\
\hline
Gyroskop & 3 * 16bit ADCs & $\pm250°/sec$, $\pm500°/sec$, $\pm1000°/sec$, $\pm2000°/sec$\\ 
\hline
Beschleunigungssensor & 3 * 16bit ADCs & $\pm2g$, $\pm4g$, $\pm8g$, $\pm16g$\\
\hline
Magnetometer & 3 * 16bit ADCs & full-scale range of $\pm$\SI{4800}{\milli\tesla\meter}T \\
\hline
Übertragung & $I^2C$, SPI, \dots & \\
\end{tabularx}
\\
\\
Der Arduino ist aufgrund seines BLE-Chips, der I²C-Verbindungsmöglichekit und der 
Output-Volt Zahl von 3,3 Volt der richtige Prozessor für dieses Projekt. Auch die 
kleinen Maße und das geringe Gewicht spielen ihm nur Pluspunkte ein.