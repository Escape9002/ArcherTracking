Die Formel für gleichmäßige Beschleunigung berechnete 
die Distanz in meinen Versuchen mit einer Genauigkeit von
+-10 cm auf 30cm Teststrecke. Wurden hierbei ebenfalls 
negative Beschleunigungen gemessen wirkten sich diese direkt
auf die Distanz aus. Dies stellte ein Problem beim abbremsen
am Ende der Teststrecke dar. Die Werte sanken wieder auf null.\\
Aus diesem Grund sind nur positive Werte für diese Funktion 
zugelassen. Hier der Code-Aussschnitt aus der Berechnung.\\
Für die Tests wurde dieser Code statt der BLE-Übetragung 
auf dem Arduino ausgeführt.

\begin{verbatim}
    if (acc > 0) {
    t = (freq / 1000); //hz is not time but frequenzy

    distance = (distance) /*+ (velocity * t) */ + (acc * (t * t) * 0.5);
    velocity = acc * t;
    }
\end{verbatim}
