Das fertige Modell ist 80 * 60 * 27 mm groß und wiegt mit den 
neuen Batterietypen ungefähr 30 Gramm. Die Größe und das Gewicht 
erfüllen die erste Anforderung. Man kann die Box überall am 
Schützen befestigen ohne das dieser durch zusätzliches Gewicht 
stark beeinflusst w"urde.\\
Die zweite Anforderung, ein geringerer Preis als Kamera, wurde erfüllt, die
Kosten belaufen sich auf 25 \euro{} pro Sensor und 150 \euro{} für ein 
Oberkörper-Tracking, bestehend aus je zwei Ober- und Unterarm sowie 
Brust- und Hüftsensor. Die letze Anforderung,
die leichte Handhabung nähert sich an, es ist mir nun möglich, Distanzen
und Winkel in eine bestimmte Richtung zu berechnen und darzustellen.
Die Auswertung der Graphen ist nach wie vor m"oglich.\\
\\
Die BLE-Übertragung funktioniert gut, das Problem der überlaufenden
Characteristics wurde behoben. Somit kann man alle Daten an das 
Smartphone senden und empfangen. Die Datenpackete bieten 
jedoch noch Platz für Verbesserung.\\
\\
Die Darstellungen der Daten funktioniert weiterhin, seit Android 11
m"ussen die gespeicherten Sch"usse manuell in den Gerätespeicher 
verschoben werden. Android richtete eine Datenschutzmaßnahme ein, 
nach der Apps nicht mehr in den Gerätespeicher, sondern in einen
appeigenen Unterordner schreiben müssen. Das Lesen stellt jedoch
kein Problem dar.\\
\\
Mit den neuen mathematischen Formeln stellt sich nun ein weiteres
Problem, die kabellose Übertragung. Je mehr Nachkommastellen übertragen
werden, desto genauer erfolgt die Berechnung der Winkel und vor allem 
der Distanz. Der Multi-Chip liefert 16 Nachkommastellen, was pro Achse
eine Characteristic benötigen w"urde, vorraussetzung, das die DLE-Funktion 
nicht aktiviert wurde.\\
Momentan sind die berechneten Ergebnisse genau genug mit nur 2 
Nachkommastellen, weshalb ich dieses Problem vorerst nicht weiter bearbeitete.\\
\\
Die Distanz stellte sich im Lauf der diesjährigen Arbeit als interessant, 
aber nicht zielführend für die Animation aus. Ersetzt wurde diese durch die funktionierende
Orientierungsberechnung. Die Berechnung und Visualisierung findet momentan 
mithilfe von Dritt-Software statt. Ein Umstieg von AppInventor auf tatsächlichen
Code wird dringend benötigt, da AppInventor weder 3D-Animationen noch die 
von mir benötigen Berechnungen anbieten oder durchführen kann.\\
Die Darstellung kann momenten nicht live angesehen werden, ein Feature, an dem noch
gearbeitet wird.\\
\\
Die berechneten Winkel und Distanzen haben einen Fehler-Wert, der im Vergleich
zu momentan benutzten Hochgeschwindigkeitskameras enorm ist, jedoch 
hat die angesprochene Kundschaft andere W"unsche.\\
\\
Geplant ist die vollständige Darstellung eines Sch"utzen in einer Android-App
und die Klassifizierung der Sch"usse via Machine-Learning. Hierzu steht ein erstes Modell,
die Datenbasis ist momentan allerdings zu gering, um belastbare Ergebnisse zu berechnen.\\
\\
\paragraph{Der Verlauf der Arbeit} gestaltete sich als komplizierter als 
gedacht. Trotz der Vorarbeit im Jahr 2020 viel es mir schwer, den dießjährigen Fokus auf die
3D-Darstellung vollständig zu erfüllen, aufgrund der oben genannten Probleme. 
Immerhin geling es mir dieses Jahr sowohl früh als auch motiviert zu arbeiten und kam einen
großen Schritt näher an den Self-Build-Bodytracker! 