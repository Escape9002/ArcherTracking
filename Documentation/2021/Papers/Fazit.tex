\chapter{Fazit}
Das fertige Modell ist 900mm*200mm*750mm groß und wiegt mit den 
neuen Batterietypen ungefähr 30 Gramm. Die Größe und das Gewicht 
erfüllen die erste Anforderung. Man kann die Box überall am 
Schützen befestigen ohne das dieser durch zusätzliches Gewicht 
stark beeinflusst w"urde.\\
Die zweite Anforderung, der geringe Preis, wurde erfüllt, die
Kosten belaufen sich auf 25 \euro{} pro Sensor. Die letze Anforderung,
die leichte Handhabung nähert sich an, es ist mir nun möglich, Distanzen
und Winkel in eine bestimmte Richtung zu berechnen und darzustellen.
Die Auswertung der Graphen ist nach wie vor m"oglich.\\
\\
Die BLE-Übertragung funktioniert gut, das Problem der überlaufenden
Characteristics wurde behoben. Somit kann man alle Daten an das 
Smartphone senden und empfangen, diese Option muss im Code nur 
aktiviert werden. Je nach Fehlerwerten der einzelnen Sensoren muss dies 
jedoch nicht getan werden.\\
\\
Die Darstellungen der Daten funktioniert weiterhin, seit Android 11
m"ussen die gespeicherten Sch"usse manuell in den Gerätespeicher 
verschoben werden. Android richtete eine Datenschutzmaßnahme ein, 
nach der Apps nicht mehr in den Gerätespeicher, sondern in einen
appeigenen Unterordner schreiben müssen. Das Lesen stellt jedoch
kein Problem dar.\\
\\
Mit den neuen mathematischen Formeln stellt sich nun ein weiteres
Problem, die kabellose Übertragung. Je mehr Nachkommastellen übertragen
werden, desto genauer erfolgt die Berechnung der Winkel und vor allem 
der Distanz. Der Multi-Chip liefert 16 Nachkommastellen, was pro Achse
eine Characteristic benötigen w"urde.\\
Momentan sind die berechneten Ergebnisse genau genug mit nur 2 
Nachkommastellen, weshalb ich dieses Problem vorerst nicht weiter bearbeitete.\\
\\
Die berechneten Winkel und Distanzen haben einen Fehler-Wert, der im Vergleich
zu momentan benutzten Hochgeschwindigkeitskameras enorm ist, jedoch 
hat die angesprochene Kundschaft andere W"unsche. Es ist m"oglich 
einen Sch"utzen mit den Rohdaten auf $\pm10$cm genau zu tracken, vielen
Sch"utzen wird diese Genauigkeit gen"ugen. F"ur eine h"ohere Genauigkeit 
k"onnen zu diesem Zeitpunkt sowohl der Code als auch die Rechnungen verbessert 
werden. Es ist noch viel Luft nach oben.\\
\\
Geplant ist die vollständige Darstellung eines Sch"utzen in der Android-App
und die Klassifizierung der Sch"usse via Machine-Learning. Hierzu m"ussen
die Rechnungen verfeinert werden und eine Datenbasis geschaffen werden.