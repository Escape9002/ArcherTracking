Die Berechnung der Distanz über Integrale ist ``The way to go''
in der Wissenschaft, obwohl die Fehler die sie mit sich bringt
bekannt sind. Eben diese Fehler wirkten sich bei meinem Sensor
stark aus.\\
Die Testergebnisse ergaben einen Fehler von +70cm auf einer Strecke
von 30cm. Wurde der Sensor zurückbewegt an seinen Startpunkt sank
der Wert jedoch wieder auf Null ab. Somit kann man schließen,
dass das Ergebniss nur falsch skaliert ist.\\
Dieser Fehler wurde noch nicht behoben. \\
Ein klarer Vorteil dieser Rechnung zeigt sich schon beim Test,
die Formel funktioniert auch für negative Beschleunigungen.
Der Wert sinkt am Ende der Teststrecke nicht auf, sondern bleibt
auf seinem hohen Wert.\\
Folgend der Code-Ausschnitt der Integral-Rechnung:
\begin{verbatim}
    t = (freq / 1000); // hz to time

    velocity = t * ((acc + accOld) / 2) + velocity;
    accOld = acc;
    distance = t * ((velocity + velocityOld) / 2) + distance;
    velocityOld = velocity;  
\end{verbatim}