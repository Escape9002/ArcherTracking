\chapter{Einleitung}
%\subsection{Problemlage}
%Als Bogenschütze bekommt man schnell mit, auf welch hohem Niveau andere Schützen es 
%schaffen, immer wieder dasselbe zu tun. Dies bezieht sich nicht nur auf den allgemeinen 
%Aufbau, bei dem diese Eigenschaft sogar von Nöten ist, sondern auch auf Fehler im Aufbau.\\
%\\
%Da ich selbst schieße und die Möglichkeit hatte in zwei unterschiedlichen Vereinen zu 
%trainieren, ergab sich schnell das Problem, dass meine Schussform drohte schlechter zu 
%werden. Ohne Trainer, der meinen persönlichen Aufbau kannte, konnte ich nicht feststellen, 
%wann ich Fehler wiederholte oder sogar neue einbaute. Eine Lösung musste her, die es mir 
%erlaubte meinen Schussablauf selbst nachzuverfolgen und Unterschiede oder Fehler selbst zu 
%finden.\\
%\\
%Bei Profi-Schützen wird schon seit langem "Ahnliches getan. Mit Hilfe von mindestens zwei 
%Hochgeschwindigkeitskameras und Punkten am Schützen, ist man in der Lage, den 
%Bewegungsablauf eines Schützen zu analysieren und dreidimensional darzustellen.\\
%\\
%Die wohl bekanntesten Aufnahmen werden derzeit vorrangig zum Perfektionieren der 
%mechanischen Ausrüstung und als Lehrmaterial verwendet. Da Profi-Schützen häufig einen 
%nahezu perfekten Aufbau haben, werden die Daten selten zum Korrigieren von Aufbaufehlern 
%eingesetzt.\\
%\\
%Zur Auswertung der Daten ist eine weitere Person von Nöten, die speziell darauf geschult 
%wurde, die erzeugten Daten auszuwerten.\\
%\\
%Die Hochgeschwindigkeitskameras, welche bei den ersten Aufnahmen von Bogenschützen 
%benutzt wurden, sind in der Lage 6000 bis 8000 Fotos in der Sekunde zu schießen. Kameras 
%dieser Qualität kosten selbst heute noch über 100 000 \euro{}.\\
%\\
%Der Raum, den diese Messmethode in Beschlag nimmt, ist ebenfalls nicht zu unterschätzen.
\section{Problemlage}
Boegnschießen, der Sport der perfekten Wiederholung. Je besser ein Schütze seinen Schussablauf wiederholen kann, 
desto einfacher ist es für den Trainer, Fehler zu finden und das Visier einzustellen. Sehr gute Trefferbilder sind 
die Folge.\\
Doch was, wenn der Trainer durch Corona-Regelungen oder Solo-Training nicht dabei sein kann? Der Schütze muss sich
selbst kontrollieren, um keine Gewöhnungsfehler einzutrainieren und schlechter zu werden. Ohne Spiegel oder andere 
Hilfsmittel ist dies jedoch kaum möglich. In genau dieser Situation befand ich mich als Schütze im Frühling 2020.\\ 
\\
Bei professionellen Schützen wird der Bewegungsablauf mittels Hochgeschwindigkeitskameras überprüft und selbst kleinste
Fehler werden entdeckt. Dabei liegt der Fokus meistens auf dem Lösvorgang. Dieser ist nur eine kleine Bewegung und 
dauert nur den Bruchteil einer Sekunde.
Benötigt wird dazu jedoch neben Kamera, Licht und Platz häufig auch eine dritte Person, die die
Datenmenge auswerten kann. All dies macht es kleinen Vereinen unmöglich auch nur an ein ähnliches System zu denken.\\
Das Ziel meiner Arbeit ist es, ein System zu entwickeln, das es dem Schützen ermöglicht, seinen eigenen Schussaufbau 
zu sehen und zuverlässig zu bewerten. Damit unterscheidet sich das Ziel meines Systems, ein Full-Body-Tracking, von der
Punktgenauen beobachtung, wie die Kamera es im obigen System beabsichtigt.

\section{Anforderungen}
Der Schütze soll seinen Schussablauf auf Grund der Daten und Visualisierung auswerten und bewerten können.
Da hierbei der gesamte Bewegungsablauf beobachtet wird, benötigt man keine zentrierte, sehr hohe Genauigkeit wie bei der 
Beobachtung des Lösens des Schützen. Somit muss die Genauigkeit der Hochgeschwindigkeitskameras nicht erreicht werden.
\\
Die Anforderungen an mein Projekt sind damit, dass eine kostengünstige Alternative 
geschaffen wird, die wenig Platz benötigt und schnelle Ergebnisse liefert. Dabei müssen diese 
Ergebnisse genau und für jeden verständlich sein.\\
Desweiteren darf der Schütze auf keinen Fall gestört werden.
Dies wirkte sich bei meiner Idee vor allem auf die Größe, das Gewicht und die 
Datenübertragung aus.\\
\\
Durch Größe und Preis soll es selbst kleinen Vereinen möglich sein, den Bogenschützen zu vermessen und gezielte Hilfe
auch ohne Trainer zu bieten. Auch Privatpersonen könnten sich so ein System sinnvoll zulegen.\\
\\
Entwickelt und gebaut wurde dieses Projekt im Schuljahr 2020/21 von Zuhause aus und wurde 
im Schuljahr 2021/22 weitergeführt.

\chapter{Vorgehensweise}
\section{Möglichkeiten}
Auf meiner Suche hatte ich die Idee, einen Ultraschallsensor am Schützen zu befestigen. Der 
Ultraschallsensor hätte am Boden befestigt werden können, um die Höhe der einzelnen 
Punkte des Schützen zu bestimmen. Ebenfalls wäre eine Kabelführung für schnellere 
Datenübertragung und damit ein günstigerer Preis möglich gewesen. Dieses statische System
hätte allerdings auf einen sich bewegenden Schützen eingestellt werden m"ussen. Ein großer Widerspruch,
lösbar nur durch weitere Technik, die zu kaufen gewesen wäre.
Auch die Verwendung einer günstigeren Kamera wäre möglich gewesen. Dabei vereint man 
allerdings alle Nachteile, die das Kamera-Tracking hat. Man braucht viel Platz und trotz sehr 
günstiger Kameras treibt man die Kosten in die Höhe.
Meine finale Idee war, eine IMU \footnote{Näheres hierzu in \textit{Kapitel 3.2 MPU9250}} und einen 
BLE-Chip\footnote{Näheres hierzu in \textit{Kapitel 4 Bluetooth-Low-Energy}} zu kombinieren. \\
Der IMU sollte mir die Beschleunigsdaten in alle Richtungen liefern, aus dehnen ich die Distanz berechnen wollte.
Als weitere Möglichkeit bot sich an, mittels des IMU die Orientierung des Sensors zu berechnen.


%\subsection{IMU vs Kameratracking}
%Mein Projekt überschneidet sich in seinen Zielen häufig 
mit Tracking das bei VR-Brillen eingesetzt wird. Hier wird
zur Feststellung der Position des Spielers häufig eine 
Kombination aus Kameratracking und Infrarot-LED. \\
Hierbei muss der Spieler die Fernbedienungen festhalten die
die Infrarot-LEDs beinhalten, während die Kameras im Raum 
so verteilt werden müssen das der Spieler immer erkannt wird.\\
Die Neigung des Kopfes und der Hände werden auch hier häufig 
Mithilfe eines IMU bestimmt.\\
\\
Da diese Systeme viel Platz benötigen, viel Geld kosten und 
für die Bildverarbeitung häufig eine große Rechenkraft benötigen
ist dieses System nicht für viele Privatnutzer sinnvoll oder 
bieten einen Bewegungsfreiraum der Sport zu lässt.\\
\\
Die IMU-Sensoren bestehen mindestens aus einem Gyroskop und einem 
Beschleunigungssensor, manche bieten sogar ein Magnetometer an.
Somit sollte es möglich sein, über die
Beschleunigung die Distantz die ein Körper mit diesem Sensor zurück
legt zu messen. Die Neigung und Orientation sind über das Gyroskop 
und Magnetometer sehr genau messbar.\\
\\
Die Vorteile der IMU liegen auf der Hand, sie sind günstig, klein
und leicht. Aus diesen Gründen trägt fast jeder heutzutage 
so einen Sensor bei sich, die meisten Handys haben ihn schon eingebaut.\\
\\
Für mein Projekt benutze ich dennoch einen eigenen IMU, um die
Qualität der Messdaten sicher zu stellen.

\section{Materialsuche}
Es galt nun, zu den genannten Kriterien die passende Hardware zu finden. Um die Bewegungen des 
Schützen nachzuverfolgen, muss ich wissen, wo die einzelnen wichtigen Punkte des Aufbaus 
sind. Dies betrifft beide Arme und die Schultern. Da sich die Arme viel bewegen, schien es mir 
möglich, mithilfe eines günstigen Beschleunigungssensors die Änderungen festzustellen. Bei 
hoher Genauigkeit könnte dieser vielleicht sogar die Bewegungen der Schultern messen.
Um den Schützen nicht zu behindern, ist eine kabellose Verbindung von Vorteil.

