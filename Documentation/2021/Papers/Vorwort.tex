\chapter[Kurzfassung]{Kurzfassung}

Mithilfe eines Bewegungssensors kann man viele 
Bewegungen erforschen und vermessen.
Mit meinem System sollen mehrere Beschleunigungssensoren 
dazu eingesetzt werden, Bewegungsabläufe aufzunehmen, 
miteinander zu vergleichen und zu erkennen.\\
\\
Die Daten werden über Bluetooth-Low-Energy an ein Handy 
geschickt, wo Sie sowohl gespeichert als auch ausgewertet 
werden können. Als Beispiel gilt hier für mich das Bogenschießen, 
bei dem selbst kleine Bewegungen immer wieder auf gleiche Weise 
ausgeführt werden müssen. Mit meinem Sensor sollen hier teure 
Kamerasysteme abgeschafft werden und es so jedem\footnote{Aus Gründen der Lesbarkeit wird das Gendern weggelassen} ermöglichen, 
selbst ohne Bogen oder Trainer bei sich Zuhause
den Bewegungsablauf zu trainieren.\\

\section{Einleitung}
\subsection{Problemlage}
Als Bogenschütze bekommt man schnell mit, auf welch hohem Niveau andere Schützen es 
schaffen, immer wieder dasselbe zu tun. Dies bezieht sich nicht nur auf den allgemeinen 
Aufbau, bei dem diese Eigenschaft sogar von Nöten ist, sondern auch auf Fehler im Aufbau.\\
\\
Da ich selbst schieße und die Möglichkeit hatte in zwei unterschiedlichen Vereinen zu 
trainieren, ergab sich schnell das Problem, dass meine Schussform drohte schlechter zu 
werden. Ohne Trainer, der meinen persönlichen Aufbau kannte, konnte ich nicht feststellen, 
wann ich Fehler wiederholte oder sogar neue einbaute. Eine Lösung musste her, die es mir 
erlaubte meinen Schussablauf selbst nachzuverfolgen und Unterschiede oder Fehler selbst zu 
finden.\\
\\
Bei Profi-Schützen wird schon seit langem "Ahnliches getan. Mit Hilfe von mindestens zwei 
Hochgeschwindigkeitskameras und Punkten am Schützen, ist man in der Lage, den 
Bewegungsablauf eines Schützen zu analysieren und dreidimensional darzustellen.\\
\\
Die wohl bekanntesten Aufnahmen werden derzeit vorrangig zum Perfektionieren der 
mechanischen Ausrüstung und als Lehrmaterial verwendet. Da Profi-Schützen häufig einen 
nahezu perfekten Aufbau haben, werden die Daten selten zum Korrigieren von Aufbaufehlern 
eingesetzt.\\
\\
Zur Auswertung der Daten ist eine weitere Person von Nöten, die speziell darauf geschult 
wurde, die erzeugten Daten auszuwerten.\\
\\
Die Hochgeschwindigkeitskameras, welche bei den ersten Aufnahmen von Bogenschützen 
benutzt wurden, sind in der Lage 6000 bis 8000 Fotos in der Sekunde zu schießen. Kameras 
dieser Qualität kosten selbst heute noch über 100 000 \euro{}.\\
\\
Der Raum, den diese Messmethode in Beschlag nimmt, ist ebenfalls nicht zu unterschätzen.

\subsection{Anforderungen}
Wenn man es nun schafft, den Kostenpunkt zu drücken und weniger Platz zu benötigen,
ergeben sich völlig neue Möglichkeiten. Dies würde vor allem den kleinen Vereinen zu Gunsten 
kommen. Sie wären endlich in der Lage, den Schussaufbau aller Schützen auf kleinste Fehler 
zu prüfen und eintrainierte Fehler einfach festzustellen.\\
\\
Die Anforderungen an mein Projekt wären damit, dass eine kostengünstige Alternative 
geschaffen wird, die wenig Platz benötigt und schnelle Ergebnisse liefert. Dabei müssen diese 
Ergebnisse genau und für jeden verständlich sein.\\
Desweiteren darf der Schütze auf keinen Fall gestört werden.
Dies wirkte sich bei meiner Idee vor allem auf die Größe, das Gewicht und die 
Datenübertragung aus.\\
Entwickelt und gebaut wurde dieses Projekt im Schuljahr 2020/21 von Zuhause aus und wurde 
im Schuljahr 2021/22 weitergeführt.

\section{Vorgehensweise}
\subsection{Möglichkeiten}
Auf meiner Suche hatte ich die Idee, einen Ultraschallsensor am Schützen zu befestigen. Der 
Ultraschallsensor hätte am Boden befestigt werden können, um die Höhe der einzelnen 
Punkte des Schützen zu bestimmen. Ebenfalls wäre eine Kabelführung für schnellere 
Datenübertragung und damit ein günstigerer Preis möglich gewesen. Dieses statische System
hätte allerdings auf einen sich bewegenden Schützen eingestellt werden m"ussen. Ein großer Widerspruch,
lösbar nur druch weitere Technik, die zu kaufen gewesen wäre.
Auch die Verwendung einer günstigeren Kamera wäre möglich gewesen. Dabei vereint man 
allerdings alle Nachteile, die das Kamera-Tracking hat. Man braucht viel Platz und trotz sehr 
günstiger Kameras treibt man die Kosten in die Höhe.
Meine finale Idee war, einen Beschleunigungssensor und einen BLE-Chip zu kombinieren.

%\subsection{IMU vs Kameratracking}
%Mein Projekt überschneidet sich in seinen Zielen häufig 
mit Tracking das bei VR-Brillen eingesetzt wird. Hier wird
zur Feststellung der Position des Spielers häufig eine 
Kombination aus Kameratracking und Infrarot-LED. \\
Hierbei muss der Spieler die Fernbedienungen festhalten die
die Infrarot-LEDs beinhalten, während die Kameras im Raum 
so verteilt werden müssen das der Spieler immer erkannt wird.\\
Die Neigung des Kopfes und der Hände werden auch hier häufig 
Mithilfe eines IMU bestimmt.\\
\\
Da diese Systeme viel Platz benötigen, viel Geld kosten und 
für die Bildverarbeitung häufig eine große Rechenkraft benötigen
ist dieses System nicht für viele Privatnutzer sinnvoll oder 
bieten einen Bewegungsfreiraum der Sport zu lässt.\\
\\
Die IMU-Sensoren bestehen mindestens aus einem Gyroskop und einem 
Beschleunigungssensor, manche bieten sogar ein Magnetometer an.
Somit sollte es möglich sein, über die
Beschleunigung die Distantz die ein Körper mit diesem Sensor zurück
legt zu messen. Die Neigung und Orientation sind über das Gyroskop 
und Magnetometer sehr genau messbar.\\
\\
Die Vorteile der IMU liegen auf der Hand, sie sind günstig, klein
und leicht. Aus diesen Gründen trägt fast jeder heutzutage 
so einen Sensor bei sich, die meisten Handys haben ihn schon eingebaut.\\
\\
Für mein Projekt benutze ich dennoch einen eigenen IMU, um die
Qualität der Messdaten sicher zu stellen.

\subsection{Materialsuche}
Es galt nun, zu den genannten Kriterien die passende Hardware zu finden. Um die Bewegungen des 
Schützen nachzuverfolgen, muss ich wissen, wo die einzelnen wichtigen Punkte des Aufbaus 
sind. Dies betrifft beide Arme und die Schultern. Da sich die Arme viel bewegen, schien es mir 
möglich, mithilfe eines günstigen Beschleunigungssensors die Änderungen festzustellen. Bei 
hoher Genauigkeit könnte dieser vielleicht sogar die Bewegungen der Schultern messen.
Um den Schützen nicht zu behindern, ist eine kabellose Verbindung von Vorteil.

