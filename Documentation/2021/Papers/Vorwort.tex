\chapter[Vorwort]{Vorwort}

Mithilfe eines Beschleunigungssensoren kann man viele 
Bewegungen erforschen und vermessen.
Mit meinem System sollen mehrere Beschleunigungssensoren 
dazu eingesetzt werden können, Bewegungsabläufe aufzunehmen, 
miteinander zu vergleichen und zu erkennen.\\
\\
Die Daten werden über Bluetooth-Low-Energy an ein Handy 
geschickt, wo Sie sowohl gespeichert als auch verwertet 
werden können. Als Beispiel gilt hier für mich das Bogenschießen, 
bei dem selbst kleine Bewegungen immer wieder auf gleiche Weise 
ausgeführt werden müssen. Mit meinen Sensor sollen hier teure 
Kamerasysteme abgeschafft werden und es so jeden ermöglichen, 
selbst ohne Bogen oder Trainer bei sich Zuhause zu den 
Bewegungsablauf zu trainieren.\\

\section{IMU vs Kameratracking}
Mein Projekt überschneidet sich in seinen Zielen häufig 
mit Tracking das bei VR-Brillen eingesetzt wird. Hier wird
zur Feststellung der Position des Spielers häufig eine 
Kombination aus Kameratracking und Infrarot-LED. \\
Hierbei muss der Spieler die Fernbedienungen festhalten die
die Infrarot-LEDs beinhalten, während die Kameras im Raum 
so verteilt werden müssen das der Spieler immer erkannt wird.\\
Die Neigung des Kopfes und der Hände werden auch hier häufig 
Mithilfe eines IMU bestimmt.\\
\\
Da diese Systeme viel Platz benötigen, viel Geld kosten und 
für die Bildverarbeitung häufig eine große Rechenkraft benötigen
ist dieses System nicht für viele Privatnutzer sinnvoll oder 
bieten einen Bewegungsfreiraum der Sport zu lässt.\\
\\
Die IMU-Sensoren bestehen mindestens aus einem Gyroskop und einem 
Beschleunigungssensor, manche bieten sogar ein Magnetometer an.
Somit sollte es möglich sein, über die
Beschleunigung die Distantz die ein Körper mit diesem Sensor zurück
legt zu messen. Die Neigung und Orientation sind über das Gyroskop 
und Magnetometer sehr genau messbar.\\
\\
Die Vorteile der IMU liegen auf der Hand, sie sind günstig, klein
und leicht. Aus diesen Gründen trägt fast jeder heutzutage 
so einen Sensor bei sich, die meisten Handys haben ihn schon eingebaut.\\
\\
Für mein Projekt benutze ich dennoch einen eigenen IMU, um die
Qualität der Messdaten sicher zu stellen.

\section{Bewegungen und Analyse}
Eine Interessante Bewegung stellt vor allem der Zugarm des 
Schüzten dar. Der Auszug verläuft nahezu linear, der häufigere Fehler 
an dieser Stelle versteckt sich allerdings in der Höhe des Zugarms. 
Um diese zu messen muss man die Erdanziehungskraft der Z-Achse herausrechnen.\\
\\
Da der Schussablauf eines Bogenschützen viele Stationen mit verschiedenen Bewegungen 
beinhaltet fällt es häufig sogar den Trainern schwer zwischen einem technisch guten
oder schlechtem Schuss zu unterscheiden.\\
Die Datenlage aus dem 9DOF-System des verwendeten MPU erzeugt eine Datenwolke, die 
diesen Vorgang fürs erste verkompliziert.\\
Die Daten sind allerdings sehr gut zu vergleichen und zu mitteln. Mit diesen 
Eigenschaften kann man über künstliche Intelligenz, genauer, Machine Learning 
die Schüsse klassifiezieren. So kann jeder Schütze seine eigene Datenbasis erstellen,
nach der sein Schuss klassifiziert und so eingeordnet werden können.\\
Ein Vergleich mit einer deutlich größeren Datenbasis als einem einzelnen Schützen ist denkbar.\\
\\
Für dieses System ist es nahezu unwichtig wie viele Körperteile überwacht werden.
Mit mehr Sensoren (oder Vergleichspunkten) wird es lediglich schwieriger für den
Schützen einen guten Wert bei der klassifiezierung zu erreichen.\\
Die Verwendung setzt natürlich vor allem korrekte, genau und viele Daten vorraus,
dies gilt für das Training des Modells so wie für die Live-Daten des Schützen.\\
\\
